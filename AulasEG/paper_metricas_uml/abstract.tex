\begin{abstract}
A utilização de diagramas de \umlS é importante na modelação de uma arquitectura de um sistema de informação, muitas vezes estes diagramas podem ser
muito completos e exaustivos de analizar, assim a análise automática de diagramas de \umlS é muito útil na medida em que pode ser um bom indicador
da qualidade e complexidade de um sistema, para além de poder dar outras informações.\\
\noindent A teoria por trás da análise automática de diagramas \umlS foi largamente desenvolvida apartir da mesma avaliação que já era feita
ao nível de código em linguagens de programação orientadas por objectos, um bom exemplo disso são as \textit{CK metrics}.\\
\noindent Aqui explicamos a investigação que existe nesta área e ainda detalhamos as expressões matemáticas que servem de base para esta análise.
\end{abstract}

