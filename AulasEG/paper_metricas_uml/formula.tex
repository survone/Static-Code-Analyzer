Temos então quatro grandes grupos de medidas que pudemos usar para definir as fórmulas que vão ser usadas para análise dos diagramas e
consequentemente extracção de uma métrica de qualidade para os diagramas UML.
\paragraph{Métricas primitivas} que consistem na extracção bruta de informaçao no que diz respeito á quantidade de métodos, classes, parametros, etc.
Temos que o número total de classes (\textit{Total Number of Classes}) é $$TNC = \sum_{i=1}^{n} tnc_i $$
o número total de relações herdadas (\textit{Total Number of Inheritance Relationships})  $$TNIR = \sum_{i=1}^{n} tnir_i $$
o número total de relações que ão sejam herdadas (\textit{ Total Number of Realization Relationships})  $$TNRR = \sum_{i=1}^{n} tnrr_i $$ por isto entende-se
uma relação entre dois elementos do modelo UML, entre dois diagramas de classes, cujo elemento cliente conhece o comportamento do outro elemento a que ele está ligado.
Temos ainda a contagem do número total de relações que existem (\textit{Total Number of Use Relationships})  $$TNUR = \sum_{i=1}^{n} tnur_i $$
o número total de associações (\textit{Total Number of Associations})    $$TNA = \sum_{i=1}^{n} tna_i $$
Uma associação representa uma relação entre duas classes. Por exemplo, numa relação entre duas classes, podemos usar associações para mostrar as decisões de design
que fizemos sobre a classe e para mostrar ainda que classe precisa dos atributos da outra.
Número total de operações (\textit{Total Number of Operation})   $$TNO = \sum_{i=1}^{n} tno_i $$
Número total de parametros (\textit{Total Number of Parameters})  $$TNP = \sum_{i=1}^{n} tnp_i $$
Número total de atributos de uma classe (\textit{Total Number of Class Attributes})  $$TNCA = \sum_{i=1}^{n} tnca_i $$

\paragraph{Métricas de propensão a falhas} (\textit{Fault-Proneness Metrics}) este grupo de métricas tem este nome porque são muito orientadas a hierarquia dos diagramas
de classes, assim sendo se houver um erro numa classe root esse erro será propagado para as classes filho que herdam.
Temos três formulas que nos ajudam a ganhar conhecimento sobre a propensão a falhas que o nosso diagrama de classes tem:
Peso de cada método por classe (\textit{Weighted Method per Class}) 
\begin{displaymath}
WMC = \sum_{i=1}^{n} c_i  \textrm{, onde $c_i$ é a complexidade dos métodos.}
\end{displaymath}

Número de sub classes que cada classe tem (\textit{Number of Children per Class})
\begin{displaymath}
NOC = \sum_{i=1}^{n} sc_i  \textrm{, onde $sc_i$ é o número de subclasses imediatas.}
\end{displaymath}
Profundidade da árvore de herança (\textit{Depth of Inheritance Tree})
\begin{displaymath}
DIT = max\_leng  \textrm{, onde $ max\_leng $ é o comprimento máximo desde a raiz até à folha.}
\end{displaymath}

\paragraph{Métricas de qualidade} (\textit{Quality Measure Metrics}) Este grupo de métricas são as mais interessantes do ponto de vista da qualidade, visto que
usam as anteriormente definidas como ferramenta para se chegar a uma conclusão sobre a qualidade do diagrama UML em questão.
Os diagramas de UML, tal como as linguagens de programação que recorrem ao modelo orientado a objectos teem a noção de protecção de elementos que definem uma class,
quer seja um método, uma variável de instância ou uma propriedade. Por isso é importante medir também este factor, assime existe o MHF (\textit{Method Hiding Factor}),
podemos dizer que esta métrica é a medida do uso de informação através de métodos e define-se por:
\begin{displaymath}
\frac{\sum_{i=1}^{rc} \sum_{m=1}^{Md(c_i)} (1-V(M_{mi}))} {\sum_{i=1}^{rc} Md(c_i)}
\end{displaymath}
onde:
$$V(M_{mi}) = \frac{\sum_{j=i}^{rc} is\_visible(M_{mi},C_j)}{TC-1} $$

\[ is\_visible(M_{mi},C_j) = \left\{ \begin{array}{l l}
	1 & \quad iff \left \{
		\begin{array}{l l}
			j \neq i \\
			C_j \texttt{may call} M_{mi}\\
		\end{array}
	\right.\\
0 & \quad otherwise\\
\end{array} \right. \]

$$TC = \textrm{Número total de classes}$$

$$Md = \textrm{Número total de métodos definidos}$$

$$V(M\_{mi}) = \textrm{A visibilidade de todas as classes onde o método $M_{mi}$ é visível}$$

\begin{description}
 \item [Quality Measure Metrics] - \begin{itemize}
								   
				   \item AHF - Attribute Hiding Factor = $ \frac{\sum_{i=1}^{rc} \sum_{m=1}^{Ad(c_i)} (1-V(A_{mi}))} {\sum_{i=1}^{rc} Ad(c_i)} $ onde: \begin{itemize}
																				      \item $V(A_{mi})$ = $\frac{\sum_{j=i}^{rc} is\_visible(A_{mi},C_j)}{TC-1} $
																				      \item $ is\_visible(A_{mi},C_j) $ = \( \left \{ \begin{array}{l l}
																									      1 & \quad iff \left \{ \begin{array}{l l}
																									                              j \neq i \\
																												      C_j \texttt{may call} A_{mi}\\
																									                             \end{array}
\right.\\
																									      0 & \quad otherwise\\
																									    \end{array} \right. \)
																				      \item TC $=$ Número total de classes
																				      \item Ad $=$ Número total de atributos definidos
																				      \item $V(A_{mi})$ $=$ A visibilidade de todas as classes onde o atributo $A_{mi}$ é visível
																				      \item AHF é então a medida do uso de informação através de atributos
																				    \end{itemize}
				   \item MIF - Method Inheritance Factor = $ \frac{\sum_{i=1}^{rc} M_i(C_i)}{\sum_{i=1}^{rc} M_a(C_i)} $ onde: \begin{itemize}
				                                                                                                                \item $ M_a(C_i) = Md(C_i) + M_i(C_i)$, é o número total de métodos disponíveis(definidos localmente mais os herdados)
				                                                                                                                \item MIF é então a medida de herdagem através de métodos
				                                                                                                               \end{itemize}
				   \item AIF - Attribute Inheritance Factor = $ \frac{\sum_{i=1}^{rc} A_i(C_i)}{\sum_{i=1}^{rc} A_a(C_i)} $ onde: \begin{itemize}
				                                                                                                                \item $ A_a(C_i) = Ad(C_i) + A_i(C_i)$, é o número total de atributos disponíveis(definidos localmente mais os herdados)
				                                                                                                                \item MIF é então a medida de herdagem através de atributos
				                                                                                                               \end{itemize}

                                  \end{itemize}
 \item [Use Case Metrics] - Para se obter complexidade dinâmica através da informação obtida de um diagrama Use Case:\begin{itemize}
                             \item NOA - Number of actors = $ \sum_{i=1}^{n} noa_i $
                             \item NOUC - Number of use cases = $ \sum_{i=1}^{n} nouc_i $
                             \item NOUCA - Use cases per Actor = $ \sum_{i=1}^{n} nouca_i $
                            \end{itemize}
  
\end{description}
