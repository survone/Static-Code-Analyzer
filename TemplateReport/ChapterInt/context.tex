\subsection{Descrição do Sistema}
O SAAP - Software para Análise e Avaliação de Programas é um sistema disponível através de uma interface web, que terá como principal função 
submeter, analisar e avaliar automaticamente programas.\\
O sistema estará disponível em parte para utilizadores não registados, mas a suas principais funcionalidades estarão apenas disponíveis para docentes 
e grupos já registados no sistema.\\
O sistema poderá ser utilizado para várias finalidades, no entanto estará direccionado para ser utilizado em concursos de programação e em elementos 
de avaliação universitários.

\subsection{Utilizadores}
Existem três entidades que podem aceder ao sistema.\\ \\
O administrador, que além de poder aceder às mesmas funcionalidades do docente, funcionalidades essas que já iremos descrever, é quem tem o poder de
criar contas para os docentes.\\
\\
O docente tem acesso a todo o tipo de funcionalidades relacionadas com a criação, edição e eliminação de concursos e enunciados, assim como consulta
de resultados dos concursos e geração/consulta de métricas para os ficheiros submetidos no sistema.\\
\\
O grupo, que pode ser constituído por um ou mais concorrentes, terá acesso aos concursos disponíveis, poderá tentar registar-se nos mesmos, e submeter
tentativas de resposta para cada um dos seus enunciados.\\
\\
Além do que já foi referido, todos os utilizadores podem editar os dados da sua conta.\\ 
Falta referir que um utilizador não registado (guest), pode criar uma conta para o seu grupo, de modo a poder entrar no sistema.

\subsection{Funcionalidades do sistema}
As várias funcionalidades do sistema já foram praticamente todas mencionadas, vamos no entanto tentar explicar a sua maioria, com um maior nível de
detalhe.

\subsubsection{Criação de contas de grupo e de docente}
\begin{itemize}
 \item Criação de conta de grupo : Qualquer utilizador não registado poderá criar uma conta no sistema para o seu grupo, através da página principal do sistema.
Terá de preencher dados referentes ao grupo e aos respectivos concorrentes.
 \item Criação de conta de docente : Como já foi referido, o administrador terá acesso a uma página onde poderá criar contas para docentes.
\end{itemize}

\subsubsection{Criação de concursos e enunciados}
Tanto o administrador como o docente podem criar concursos. Depois de preencher todos os dados do concurso podem passar à criação de enunciados.
Nesta altura caso prefiram, podem importar enunciados previamente criados em xml.
O concurso só ficará disponível na data definida aquando da criação do concurso.

\subsubsection{Registo e participação nos concursos}
Um grupo que esteja autenticado no sistema pode tentar registar-se num dos concursos disponíveis, e será bem sucedido se a chave que utilizar for a correcta.\\
Depois de se registar no concurso, o tempo para a participação no mesmo inicia a contagem decrescente e o grupo poderá começar a submeter tentativas para
cada um dos enunciados. O sistema informará, pouco depois da submissão, se a resposta estaria correcta ou não.
Depois do tempo esgotar o grupo não pode submeter mais tentativas.

\subsubsection{Geração das métricas para os ficheiros submetidos}
O docente ou o administrador a qualquer altura podem pedir ao sistema que gere as métricas para as tentativas submetidas.

\subsubsection{Consulta dos ficheiros com as métricas e dos resultados do concurso}
O docente pode aceder aos logs que contêm a informação sobre as tentativas submetidas, ou se desejar, apenas aceder a informações mais específicas, tal como 
qual exercício tentou submeter determinado grupo, e se teve sucesso ou não.\\
Pode também visualizar os ficheiros com as informações das métricas.

\subsubsection{Ferramentas}
base de dados - DB2?\\
scripts perl\\
aplicaçao web - ruby on rails\\



