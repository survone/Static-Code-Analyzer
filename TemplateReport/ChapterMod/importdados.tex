Uma das funcionalidades requeridas ao nosso sistema é a importação de enunciados e tentativas no formato xml.
Esta funcionalidade será bastante útil para demonstrar e testar o sistema, sem que se tenha de criar manualmente os enunciados
 usando a interface gráfica, ou se tenha que submeter ficheiros com código fonte, de modo a serem geradas tentativas.
Os campos presentes no xml de cada uma das entidades, \textit{enunciado} e \textit{tentativa}, são praticamente os mesmos 
que estão descritos no modelo de dados das respectivas entidades.\\
\\
No xml do enunciado, não há nada muito relevante a acrescentar, além de que não contém uma id para o enunciado, pois esta será gerada 
automaticamente pelo sistema. Passamos agora a apresentar um exemplo do mesmo.

%\lstinputlisting{ChapterMod/importdadosFiles/enunciado.xml}


Quanto ao xml para a \textit{tentativa} há que realçar o facto de que o código fonte do programa vai dentro de uma tag xml.  Além da tag xml, o código fonte
terá de ir cercado de uma secção CDATA. Isto acontece para que o que o código fonte não seja processado com o restante xml que o contém.\\
\\

%\lstinputlisting{ChapterMod/importdadosFiles/tentativa.xml}

Para que todos os dados contidos nos ficheiros xml possam ser facilmente validados, foram criados dois \textit{XML schema}.
Neste schema definimos quais as tags que devem existir em cada xml, o tipo de dados e até a gama de valores que serão contido por cada tag e
 a multiplicidade das tags .\\
\\
Nesta fase inicial do projecto ainda não foram sempre especificados  os tipos de dados que serão contidos por cada tag.\\
No entanto, para alguns dos casos em que tal aconteceu apresentaremos alguns exemplos e explicações.\\
\\
No xsd referente ao \textit{enunciado} encontramos o elemento \textit{Peso}, que é um exemplo de uma tag que contém restrições.
O \textit{Peso} terá de ser um inteiro e terá um valor entre 0 e 100.
Já o elemento \textit{Linguagem} é também restringido, mas de uma forma ligeiramente diferente. A \textit{Linguagem} será uma string, mas
apenas poderá tomar um dos valores enumerados no xsd.\\
\\
%\lstinputlisting{ChapterMod/importdadosFiles/enunciado.xsd}

No xsd para a \textit{tentativa} podemos evidenciar a multiplicidade das tags, ou seja, quantas vezes algumas delas se podem repetir.
Na \textit{tentativa}, existe um \textit{Dict}, que contém uma ou mais tags \textit{Teste}.
Para definirmos que possam existir mais de que uma tag \textit{Teste} dentro de \textit{Dict}, adicionamos o atributo \textit{maxOccurs},
na entidade \textit{Teste}, com o valor \textit{``unbounded''}. O valor mínimo não é necessário definir, porque é um por default.


%\lstinputlisting{ChapterMod/importdadosFiles/enunciado.xsd}

