\chapter{Introdução} \label{chap int}
Este relatório descreve o projecto desenvolvido para o módulo Projecto Integrado da UCE de Engenharia Linguagens do Mestrado em Engenharia Informática da Universidade do Minho.\\

Pretende-se que este projecto seja um \emph{Software} para Análise e Avaliação de Programas (SAAP), tendo este \emph{software} como objectivo criar um 
ambiente de trabalho, com um interface Web, que permita a docentes/alunos avaliar/submeter programas automaticamente.

\section{Motivação}
Este trabalho é muito interessante do ponto de vista de produto final, como de obra didáctica para a sua concretização. Para a sua feitura os autores terão de aplicar
conhecimentos adquiridos na área de arquitectura de um sistema de informação, desenvolvimento para a web, lingugaens de scripting, bases de dados, processamento de textos, etc\ldots\\
Como se pode facilmente concluir, este é um projecto que do ponto de vista técnico é muito motivador para nós devido à sua magnitude e inovação técnica que implica aos seus autores.

\section{Objectivos}
Este projecto tem como objectivos consolidar conhecimentos adquiridos nos diferentes módulos da UCE de Engenharia de Linguagens
Assim sendo, vamos recorrer a tecnologia aprendida durante as aulas, bemc omo tecnologia que já conheciamos e aprendemos durante a nossa formação académica ou à parte desta.
Um dos nossos grandes objectivos pessoais é consolidar e desenvolver ainda mais o uso de tecnologias variadas, para arranjar uma solução elegante, funcional e que cumpra os requesitos do sistema descrito.
Assim, este projecto é mais do que mais um projecto de mestrado, passando a ser encarado como um desafio ao conhecimento e aa por em prática conhecimentos adquiridos.

\subsection{Ferramentas}
As ferramentas/tecnologias que pensamos ir precisar são as seguinte:
\begin{itemize}
\item DB2
\item Perl
\item Ruby on Rails
\item Haskell
\end{itemize}

Iremos usar DB2 por suportar nativamente XML e ainda XPath e XQuery para fazer travessias no XML. O uso de Perl prende-se com o facto de ser incrivelmente fácil criar
sem muito esforço pequenas ferramentas que acreditamos serem uteis para identificar padrões em texto ou cortar pequenos pedaços de texto.\\
O uso de RoR deve-se ao facto da simplicidade em criar ambientes web. Decidimos ainda utilizar Haskell eventualmente em tarefas mais complexas, por ser uma linguagem de rápida
implementação e bastante segura.


\section{Contextualização}
\subsection{Descrição do Sistema}
O SAAP - Software para Análise e Avaliação de Programas é um sistema disponível através de uma interface web, que terá como principal função 
submeter, analisar e avaliar automaticamente programas.\\
O sistema estará disponível em parte para utilizadores não registados, mas a suas principais funcionalidades estarão apenas disponíveis para docentes 
e grupos já registados no sistema.\\
O sistema poderá ser utilizado para várias finalidades, no entanto estará direccionado para ser utilizado em concursos de programação e em elementos 
de avaliação universitários.

\subsection{Utilizadores}
Existem três entidades que podem aceder ao sistema.\\ \\
O administrador, que além de poder aceder às mesmas funcionalidades do docente, funcionalidades essas que já iremos descrever, é quem tem o poder de
criar contas para os docentes.\\
\\
O docente tem acesso a todo o tipo de funcionalidades relacionadas com a criação, edição e eliminação de concursos e enunciados, assim como consulta
de resultados dos concursos e geração/consulta de métricas para os ficheiros submetidos no sistema.\\
\\
O grupo, que pode ser constituído por um ou mais concorrentes, terá acesso aos concursos disponíveis, poderá tentar registar-se nos mesmos, e submeter
tentativas de resposta para cada um dos seus enunciados.\\
\\
Além do que já foi referido, todos os utilizadores podem editar os dados da sua conta.\\ 
Falta referir que um utilizador não registado (guest), pode criar uma conta para o seu grupo, de modo a poder entrar no sistema.

\subsection{Funcionalidades do sistema}
As várias funcionalidades do sistema já foram praticamente todas mencionadas, vamos no entanto tentar explicar a sua maioria, com um maior nível de
detalhe.

\subsubsection{Criação de contas de grupo e de docente}
\begin{description}
 \item[Criação de conta de grupo:] Qualquer utilizador não registado poderá criar uma conta no sistema para o seu grupo, através da página principal do sistema.
Terá de preencher dados referentes ao grupo e aos respectivos concorrentes.
 \item[Criação de conta de docente:] Como já foi referido, o administrador terá acesso a uma página onde poderá criar contas para docentes.
\end{description}

\subsubsection{Criação de concursos e enunciados}
Tanto o administrador como o docente podem criar concursos. Depois de preencher todos os dados do concurso podem passar à criação de enunciados.
Nesta altura caso prefiram, podem importar enunciados previamente criados em xml.
O concurso só ficará disponível na data definida aquando da criação do concurso.

\subsubsection{Registo e participação nos concursos}
Um grupo que esteja autenticado no sistema pode tentar registar-se num dos concursos disponíveis, e será bem sucedido se a chave que utilizar for a correcta.\\
Depois de se registar no concurso, o tempo para a participação no mesmo inicia a contagem decrescente e o grupo poderá começar a submeter tentativas para
cada um dos enunciados. O sistema informará, pouco depois da submissão, se a resposta estaria correcta ou não.
Depois do tempo esgotar o grupo não pode submeter mais tentativas.

\subsubsection{Geração das métricas para os ficheiros submetidos}
O docente ou o administrador a qualquer altura podem pedir ao sistema que gere as métricas para as tentativas submetidas.

\subsubsection{Consulta dos ficheiros com as métricas e dos resultados do concurso}
O docente pode aceder aos logs que contêm a informação sobre as tentativas submetidas, ou se desejar, apenas aceder a informações mais específicas, tal como 
qual exercício tentou submeter determinado grupo, e se teve sucesso ou não.\\
Pode também visualizar os ficheiros com as informações das métricas.

\section{Estrutura do Relatório}
Estrutura do Relatório...
