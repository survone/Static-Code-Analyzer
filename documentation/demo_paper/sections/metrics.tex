\newcommand{\metricsins}{\mathbin{>\mkern-7mu\circ\mkern-9mu>}}
\newcommand{\metricscat}{\mathbin{>\mkern-7mu+\mkern-11mu>}}

\section{Metrics and Assessment}\label{metrics}
\subsection{API design}
To be able to manipulate and store the calculated metrics we have develop an API. Here are the relevent data types:
\begin{haskell*}
  \hskwd{data} Metrics
    & = & Metrics (Map MetricName MetricValue)\\
  \hskwd{type} MetricName
    & = & (Name,FileName,FunctionName)\\
  \hskwd{data} MetricValue
    & = & Num Double\\
    & | & Clone (Map FileDst [(Ocurrency, LineSrc, LineDst)])\\
    & | & Includes ([SystemIncludes],[Includes])
\end{haskell*}

\noindent We have used a Map to store our metrics, where the key is the triple $MetricName$. Some metrics just make sense regarding all the software package,
others may only be relevant regarding files and other may be extracted from a function. By letting some of empty fields in the triple we express this three categories
of metrics. Regarding the $MetricValue$ we can have multiple types of metrics, most of them just return a numerary value, the cosntructor $Num$. And then we have more specific
metrics related to clones found in the code, list of include and system includes.\\
\indent Now we present one of the most important function to append metrics. The power of \textit{Haskell} infix notation allow us to use comfortably this kind of functions.

\begin{haskell*}
  \metricsins ::  Metrics & \to& (MetricName, &MetricValue) \to Metrics\\
  m \metricsins (mn,mv) 
     &\mid &member mn m & = \hslet{fm  & = & fromMetrics m\\(Just mv') & = & lookup mn fm}
	{\hskwd{if} mv' \equiv mv \hskwd{then} m \hskwd{else} unpackPack ins m}\\
                            &\mid &otherwise   & = unpackPack ins m
	\hswhere{ins = insert mn mv}
\end{haskell*}

The previous function can be used when we want create a new metric, if so we must use $emptyMetrics$ as first paramenter.
The $unpackPack$ function used is a trick to remove the cosntructor, apply some function and return $Metrics$, could be defined as: $unpackPack~f \doteq toMetrics  \circ f \circ fromMetrics$.\\
\indent The following function can be used when the user already have some metrics calculated and want to concat them into one.

\begin{haskell*}
\metricscat & :: & Metrics \to Metrics \to Metrics\\
m1 \metricscat m2 & = & toMetrics (union (fromMetrics m1) (fromMetrics m2))
\end{haskell*}

As an example we can use this function to take a list of $Metrics$ and concat all of them into just one $Metrics$ value: $foldl~(\metricscat)~emptyMetrics$.

\subsection{Metrics}
Write about the implemented metrics, formulae, etc...
