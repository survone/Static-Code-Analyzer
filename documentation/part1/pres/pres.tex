\documentclass{beamer}

\mode<presentation>
{
   \usetheme{EEng}
%   \usetheme{Warsaw}
  \setbeamercovered{transparent}
  \setbeamercolor{background canvas}{bg=black!0}
}

\usepackage{enumerate}
\usepackage{graphics}
\usepackage{ucs}
\usepackage[utf8x]{inputenc}
\usepackage[english]{babel}
\usepackage{amsmath}
\usepackage{amsfonts}
\usepackage{xcolor}
\usepackage{pgf}
\usepackage{hyperref}
\usepackage{url}
\usepackage{multicol}   % add-on
\usepackage{boxedminipage} 
\usepackage{indentfirst}   % add-on
\usepackage{float}
\usepackage[all]{xypic}
\usepackage{listings}
\usepackage{verbatim}
\usepackage{boxedminipage}

% % % inicio do listings e ref
\definecolor{darkblue}{rgb}{0,0,0.6}
\definecolor{gray_ulisses}{gray}{0.55}
\definecolor{castanho_ulisses}{rgb}{0.71,0.33,0.14}
\definecolor{preto_ulisses}{rgb}{0.41,0.20,0.04}
\definecolor{green_ulises}{rgb}{0.2,0.75,0}

\hypersetup{
	a4paper,
	pdftex,
	bookmarks,
	colorlinks,
    citecolor=darkblue,
    linkcolor=darkblue,
    urlcolor=darkblue,
    filecolor=darkblue
}

\lstdefinelanguage{TclUlisses}
{
		language=tcl,
        basicstyle=\ttfamily\scriptsize,
        sensitive=true,
        morecomment=[l][\color{gray_ulisses}\ttfamily\scriptsize]{\#},
        morestring=[b]",
        stringstyle=\color{red},
        showstringspaces=false,
        numberstyle=\tiny,
        numberblanklines=true,
        showspaces=false,
        breaklines=true,
        showtabs=false,
        emph=
        {[1]
                join,reverse,split,take,tail,drop,theorem
        },
        emphstyle={[1]\color{blue}},
        emph=
        {[2]
                Bit,inf,fin
        },
        emphstyle={[2]\color{castanho_ulisses}},
        emph=
        {[3]
                case,class,data,deriving,do,else,if,import,in,infixl,infixr,instance,let,
                module,of,primitive,then,type,where
        },
        emphstyle={[3]\color{preto_ulisses}\textbf},
        emph=
        {[4]
                True,False
        },
        emphstyle={[4]\color{castanho_ulisses}\textbf},
}

\lstdefinelanguage{Terminall} {
       basicstyle=\scriptsize\ttfamily,
       breaklines=true,
       breakautoindent=false,
       showstringspaces=false
}
\lstdefinelanguage{Terminal} {
       basicstyle=\tiny\ttfamily,
       breaklines=true,
       breakautoindent=false,
       showstringspaces=false
}
% % % fim do listings e ref

% % % inicio da definicao de comandos
\newcommand{\ixload}{\emph{IxLoad}}
\newcommand{\ixos}{\emph{IxOS}}
\newcommand{\ats}{\emph{ATS}}
\newcommand{\autoeasy}{\emph{AutoEASY}}
\newcommand{\ixia}{\emph{Ixia}}
\newcommand{\http}{\emph{HTTP}}
\newcommand{\ftp}{\emph{FTP}}
\newcommand{\ssl}{\emph{SSL}}
\newcommand{\pop}{\emph{POP$3$}}
\newcommand{\smtp}{\emph{SMTP}}
\newcommand{\fpm}{\emph{FPM}}
\newcommand{\sfpm}{\emph{SFPM}}
\newcommand{\ffpm}{\emph{FFPM}}
\newcommand{\cisco}{\emph{Cisco}}
% % % fim da definicao de comandos

\title{\sfpm~Threat Mitigation}
\author{Ulisses Ara\'ujo Costa (ucosta)}
\date{\today}
\institute{Cisco Systems Inc.}

\AtBeginSubsection[] {
  \begin{frame}<beamer>
    \frametitle{Index}
    \scriptsize{\tableofcontents[currentsection,currentsubsection]}
  \end{frame}
}

\AtBeginSection[] {
  \begin{frame}<beamer>
    \frametitle{Index}
    \scriptsize{\tableofcontents[currentsection]}
  \end{frame}
}
\begin{document}
\begin{frame}
   \titlepage
\end{frame}

%IxLoad is a highly scalable solution to accurately assess the performance of content-aware devices and networks. IxLoad creates
%real-world traffic scenarios at the TCP (Layer 4) and application layers (Layer 7) by emulating clients and servers for a variety of
%protocols including HTTP, SSL, FTP, Mail (SMTP, POP3) and Streaming (RTP, RTSP) protocols. IxLoad utilizes Ixia's purpose-built
%hardware including the Application Load Module (ALM) and the TXS family of Ethernet Load Modules. Each port of these Load Modules has a
%CPU running the Linux operating system with a standards-compliant TCP/IP protocol stack and can emulate thousands of users.

\section{Introduction to Testing}
\pgfdeclareimage[width=.1\textwidth]{tcl}{images/tcl}
\pgfdeclareimage[width=.2\textwidth]{solaris}{images/solaris}
\pgfdeclareimage[width=.1\textwidth]{passfail}{images/green_check_red_cross}
\pgfdeclareimage[width=.8\textwidth]{topo}{images/topo}

\begin{frame} \frametitle{Tests Jargon}
\begin{itemize}
    \item ATS - Automated Test Solutions
    \item \href{http://earms-app.cisco.com/server/earms}{eARMS - Extended Automated Regression Management System}\\
    \item \href{http://tft.cisco.com/run-search.cmd?search=8607691}{TFT - Test Feature Tracker}\\
    \item \href{http://tims.cisco.com/browsing.cmd?proj=91\&folder=95370}{TIMS - Test Information Management System}\\
    \item \href{http://earms-trade.cisco.com/tradeui/dashboardviewer.faces?ats=/auto/stg-devtest/ucosta/ats5.1.0}{TRADe - Test Results Analysis and Debugging}\\
    \item Testbed
\end{itemize}
\begin{center}
	 \pgfuseimage{topo}
\end{center}
\end{frame}

\begin{frame} \frametitle{\ats~Setup}
\xyoption{matrix}
\xymatrix{ &\pgfuseimage{tcl}  \ar@/^/[rd]^{} \ar@/^/[ld]_{}& \\
           \boxed{CONFIG~file} \ar@/^/[rd]^{autoeasy}&  &\boxed{.job~file} \ar@/^/[ld]_{autoeasy} \\
           & \pgfuseimage{solaris} &
         }
\begin{block}{}
\begin{itemize}
\item Scripts written in Tcl
\item CONFIG file with topology of the testbed (plus cleanup and setup)
\item .job file with all calls to the scripts
\end{itemize}
\end{block}
\end{frame}

\begin{frame} \frametitle{Management of Scripts and Results}
\xyoption{matrix}
\xymatrix{ \boxed{makefile} \ar@/^/[rd]^{make}&  &\boxed{\href{http://earms-app.cisco.com/server/earms}{eARMS}} \ar@/^/[ld]_{run} \\
           \boxed{CVS}& \ar@/^/[ld]^{report} \ar@/^/[l]^{commit} \pgfuseimage{tcl} \pgfuseimage{tcl}  \ar@/^/[rd]_{report} \ar@/^/[r]_{production} &\boxed{\href{http://tft.cisco.com/run-search.cmd?search=8607691}{TFT}}\\
           \boxed{\href{http://earms-trade.cisco.com/tradeui/dashboardviewer.faces?ats=/auto/stg-devtest/ucosta/ats5.1.0}{TRADe}} \pgfuseimage{passfail} & &  \pgfuseimage{passfail} \boxed{\href{http://tims.cisco.com/browsing.cmd?proj=91\&folder=95370}{TIMS}}
         }
\end{frame}

\section{Sessin-based Flexible Packet Matching (SFPM)}
\begin{frame}[fragile] \frametitle{\fpm~vs \sfpm}
    \begin{block}{Flexible Packet Matching (\fpm)}
        \begin{itemize}
            \item Current \fpm~is a stateless per packet classification mechanism.
            \item \fpm~works well when the filter information exists in all packets of a flow.
            \item However, \fpm~can only apply actions to the those packets, and miss the rest of the packets in the same flow.
        \end{itemize}
    \end{block}

    \begin{block}{Session-based Flexible Packet Matching (\sfpm)}
        \begin{itemize}
            \item \sfpm~allows customers to create their own filtering policies that can immediately detect and block attacks.
            \item Session-based \fpm~allows session-based classification and actions.
        \end{itemize}
    \end{block}
\end{frame}

\begin{frame}[fragile] \frametitle{Configuration Example - Action}
\begin{lstlisting}[language=Terminal]
router(config)#load fpm
Try to load bundle PHDF files ...
router(config)#class-map type access-control match-all c1
router(config-cmap)#match field TCP source-port eq 1024
router(config-cmap)#class-map type access-control match-any c2
router(config-cmap)#match start TCP payload-start offset 0 size 5 regex "GET /"
router(config-cmap)#policy-map type access-control p1
router(config-pmap)#class c1
router(config-pmap-c)#log all
router(config-pmap-c)#class c2
router(config-pmap-c)#log all
router(config-pmap-c)#policy-map type access-control fpm1
router(config-pmap)#class ip_tcp_stack
router(config-pmap-c)#service-policy p1
router(config-pmap-c)#interface FastEthernet0/1
router(config-int)#service-policy type access-control input fpm1
\end{lstlisting}
\begin{block}{}
\begin{itemize}
	\item Match TCP source-port number
    \item Match TCP payload regular expression
	\item log the sessions
	\item Attach the policy to the interface
\end{itemize}
\end{block}
\end{frame}

\begin{frame}[fragile] \frametitle{Configuration Example - Nested}
\begin{lstlisting}[language=Terminal]
router(config)#load fpm
Try to load bundle PHDF files ...
router(config)#class-map type access-control match-all c1
router(config-cmap)#match field ICMP type eq 8
router(config-cmap)#class-map type access-control match-all c2
router(config-cmap)#match field ICMP checksum eq 123456
router(config-cmap)#class-map type access-control match-all c3
router(config-cmap)#match class c1 session
router(config-cmap)#policy-map type access-control p1
router(config-pmap)#class c3
router(config-pmap-c)#drop all
router(config-pmap-c)#policy-map type access-control fpm1
router(config-pmap)#class ip_icmp_stack
router(config-pmap-c)#service-policy p1
router(config-pmap-c)#interface FastEthernet0/1
router(config-if)#service-policy type access-control input fpm1
\end{lstlisting}
\begin{block}{}
\begin{itemize}
	\item Match ICMP type
    \item Match ICMP checksum
	\item drop the sessions
	\item Attach the policy to the interface
\end{itemize}
\end{block}
\end{frame}

\begin{frame}[fragile] \frametitle{Configuration Example - Session Packet Range}
\begin{lstlisting}[language=Terminal]
router(config)#load fpm
Try to load bundle PHDF files ...
router(config)#class-map type access-control match-all c2
router(config-cmap)#match field TCP source-port eq 1024
router(config-cmap)#class-map type access-control match-all c3
router(config-cmap)#$ TCP payload-start offset 0 size 5 regex "GET /"        
router(config-cmap)#class-map type access-control match-all c1
router(config-cmap)#match class c3 packet-range 3 4
router(config-cmap)#policy-map type access-control p1
router(config-pmap)#class c1
router(config-pmap-c)#log all
router(config-pmap-c)#policy-map type access-control fpm1
router(config-pmap)#class ip_tcp_stack
router(config-pmap)#service-policy p1
router(config-pmap-c)#interface FastEthernet0/1
router(config-if)#service-policy type access-control input fpm1
\end{lstlisting}
\begin{block}{}
\begin{itemize}
	\item Match TCP source-port
    \item Match TCP regexp (HTTP)
	\item log all sessions that have this match between packet 3 and 4
	\item Attach the policy to the interface
\end{itemize}
\end{block}
\end{frame}

\begin{frame}[fragile] \frametitle{\sfpm~Demo}
\begin{center}
	 \huge{\sfpm~Demo}
\end{center}
\end{frame}

\section{SFPM Tests}
\subsection{Testcases}
\begin{frame}[fragile] \frametitle{TCP/UDP/ICMP testcases - case1\_config}
\begin{enumerate}
    \item Add a filter to existing class-map
    \item Remove then add a new filter to existing class-map
    \item Add a SFPM action for class-map
    \item Remove action from class-map
    \item Add class-map to using policy-map
    \item Remove class-map from policy-map
    \item Add child class-map in stack class-map
    \item Remove child class-map from stack class-map
    \item Remove child policy-map
    \item Remove parent policy-map
\end{enumerate}
\end{frame}

\begin{frame}[fragile] \frametitle{TCP/UDP/ICMP testcases - case2\_config}
\begin{enumerate}
    \item Add nested class session into class-map
    \item Remove nested class session from class-map
    \item Add filter in nested class
    \item Remove filter from nested class
    \item Add action into nested class
    \item Remove action from nested class
    \item Remove parent class-map (contains nested class) in policy-map
    \item Add parent class-map (contains nested class) in policy-map
    \item Remove child policy-map (contains nested class)
    \item Remove parent policy-map attached to interface
    \item Create consecutive nested class-map
    \item Create circular nested class
\end{enumerate}
\end{frame}

\begin{frame}[fragile] \frametitle{TCP/UDP/ICMP testcases - case3\_config}
\begin{enumerate}
    \item Add nested class session into class-map
    \item Remove nested class session from class-map
    \item Add filter in nested class
    \item Remove filter from nested class
    \item Add action into nested class
    \item Remove action from nested class
    \item Remove parent class-map (contains nested class) in policy-map
    \item Add parent class-map (contains nested class) in policy-map
    \item Remove child policy-map (contains nested class)
    \item Remove parent policy-map attached to interface
    \item Create consecutive nested class-map
    \item Create circular nested class
    \item Check packet range number
\end{enumerate}
\end{frame}
\begin{frame}[fragile] \frametitle{TCP/UDP/ICMP testcases - Case Ingress}
\begin{lstlisting}[language=Terminall]
                      +------------+
               fpm1   |            |
  Client  ... =======>|            |===== ... Server
                  G0/3|    U U T   |
                      |            |
                      |            |
                      +------------+
\end{lstlisting}
\end{frame}
	
\begin{frame}[fragile] \frametitle{TCP/UDP/ICMP testcases - Case Egress}
\begin{lstlisting}[language=Terminall]
                      +------------+
                      |            | fpm1
  Client  ... ========|            |=====> ... Server
                      |    U U T   |G0/2
                      |            |
                      |            |
                      +------------+
\end{lstlisting}
\end{frame}

\begin{frame}[fragile] \frametitle{TCP/UDP/ICMP testcases - Case Ingress$+$Egress}
\begin{lstlisting}[language=Terminall]
                      +------------+
               fpm1   |            | fpm1
  Client  ... =======>|            |====> ... Server
                G0/3  |    U U T   |G0/2
                      |            |
                      |            |
                      +------------+
\end{lstlisting}
\end{frame}

\begin{frame}[fragile] \frametitle{TCP/UDP/ICMP testcases - Case Input$+$Output}
\begin{lstlisting}[language=Terminall]
                      +------------+
         |            |            |
         |     fpm1   |            |
         |... =======>|            |
         |      G0/3  |    U U T   |
  Client |            |            |===== ... Server
         |     fpm3   |            |
         |... <=======|            |
         |       G0/3 |            |
         |            |            |
                      +------------+
\end{lstlisting}
\end{frame}

\begin{frame}[fragile] \frametitle{Testcases - CEF/Process}
    \begin{block}{Cisco Express Forwarding}
        \begin{itemize}
            \item Cisco's Express Forwarding (CEF) is an advanced, Layer 3 switching technology inside a router. It defines the fastest method by which a Cisco router forwards packets from ingress to egress interfaces.
            \item Process switching uses the CPU on every packet, CEF only needs to the CPU for the first packet of each session.
        \end{itemize}
    \end{block}
\end{frame}

\begin{frame}[fragile] \frametitle{TCP/UDP/ICMP testcases - case1\_$<$config$|$traffic$>$\_$<$cef$|$process$>$\_$<$TCP$|$UDP$|$ICMP$>$}
Action policies with log all as action
\begin{itemize}
    \item case1 config
	\item case1\_traffic\_cef\_TCP
	\item case1\_traffic\_process\_TCP
	\item case1\_traffic\_cef\_UDP
	\item case1\_traffic\_process\_UDP
	\item case1\_traffic\_cef\_ICMP
	\item case1\_traffic\_process\_ICMP
\end{itemize}
\end{frame}

\begin{frame}[fragile] \frametitle{TCP/UDP/ICMP testcases - case1\_2\_traffic\_$<$cef$|$process$>$\_$<$TCP$|$UDP$|$ICMP$>$}
Nested policies with log as action
\begin{itemize}
	\item case1\_2\_traffic\_process\_TCP
	\item case1\_2\_traffic\_cef\_TCP
	\item case1\_2\_traffic\_process\_UDP
	\item case1\_2\_traffic\_cef\_UDP
	\item case1\_2\_traffic\_process\_ICMP
	\item case1\_2\_traffic\_cef\_ICMP
\end{itemize}
\end{frame}

\begin{frame}[fragile] \frametitle{TCP/UDP/ICMP testcases - case2\_$<$config$|$traffic$>$\_$<$cef$|$process$>$\_$<$TCP$|$UDP$|$ICMP$>$}
Nested policies with log all as action
\begin{itemize}
    \item case2\_config
	\item case2\_traffic\_cef\_TCP
	\item case2\_traffic\_process\_TCP
    \item case2\_traffic\_cef\_UDP
    \item case2\_traffic\_process\_UDP
    \item case2\_traffic\_cef\_ICMP
    \item case2\_traffic\_process\_ICMP
\end{itemize}
\end{frame}

\begin{frame}[fragile] \frametitle{Multiple Flows testcases}
\begin{block}{}
    For each testcase send multiple-flows TCP/UDP/ICMP traffic
\end{block}
\begin{itemize}
    \item action\_multiple\_flow\_process
	\item action\_multiple\_flow\_cef
	\item nested\_multiple\_flow\_cef\_log
    \item nested\_multiple\_flow\_cef\_logAll
    \item nested\_multiple\_flow\_process\_log
    \item nested\_multiple\_flow\_process\_logAll
\end{itemize}
\end{frame}

\begin{frame}[fragile] \frametitle{Change Configuration testcases - Change config}
\begin{itemize}
    \item action\_change\_config\_cef
    \item action\_change\_config\_process
    \item nested\_change\_config\_cef
    \item nested\_change\_config\_process
\end{itemize}
\begin{block}{Method}
\begin{enumerate}
    \item Create a new policy
    \item Send TCP traffic
    \item In the midlle of traffic sending change the policies
\end{enumerate}
\end{block}
\end{frame}

\begin{frame}[fragile] \frametitle{Change Configuration testcases - Apply config}
\begin{itemize}
    \item action\_apply\_config\_cef
    \item action\_apply\_config\_process
    \item nested\_apply\_config\_cef
    \item nested\_apply\_config\_process
\end{itemize}
\begin{block}{Method}
\begin{enumerate}
    \item Delete all the policies
    \item Send TCP traffic
    \item In the midlle of traffic sending apply the policies
\end{enumerate}
\end{block}
\end{frame}

\begin{frame}[fragile] \frametitle{Change Configuration testcases - Delete config}
\begin{itemize}
    \item action\_delete\_config\_cef
    \item action\_delete\_config\_process
    \item nested\_delete\_config\_cef
    \item nested\_delete\_config\_process
\end{itemize}
\begin{block}{Method}
\begin{enumerate}
    \item Create a new policy
    \item Send TCP traffic
    \item In the midlle of traffic sending delete the policies
\end{enumerate}
\end{block}
\end{frame}

\begin{frame}[fragile] \frametitle{Bugs}
\begin{description}
    \item[\href{http://cdets.cisco.com/apps/dumpcr?identifier=CSCtg61173\&content=summary\&format=html}{CSCtg61173}] UDP classification fails for output direction in cef
    \item[\href{http://cdets.cisco.com/apps/dumpcr?identifier=CSCtg60872\&content=summary\&format=html}{CSCtg60872}] Regex classification is not working in TCP traffic with input+output
    \item[\href{http://cdets.cisco.com/apps/dumpcr?identifier=CSCtg61221\&content=summary\&format=html}{CSCtg61221}] SFPM (FFPM) Stateful classification in input and output direction
\end{description}
\end{frame}

\subsection{Traffic generators}
\begin{frame}[fragile] \frametitle{Pagent Make your NET{\color{red}WORK}}
PAGENT is an IOS Based Testing Tool used to generate and capture, emulate large routed networks, and generate session based traffic.\\
The test tools are included in special IOS Pagent images.
\begin{block}{}
\begin{description}
    \item[TGN] Traffic Generator - generates TCP/UDP/ICMP traffic
    \item[HTTPSE] HTTP Session Emulator
    \item[PKTS] Packet Count and Capture
\end{description}
\end{block}
\end{frame}

\section{Performance Tests}
\begin{frame}[fragile] \frametitle{Why \ixload?}
\begin{block}{}
\begin{itemize}
	\item Works on L4 and up
	\item Creates real-world traffic scenarios
	\item Emulate clients and servers of \http, \ssl, \ftp, \smtp, \pop
\end{itemize}
\end{block}
\end{frame}

\pgfdeclareimage[width=.9\textwidth]{solarispcixia}{images/unix_pc_ixia}
\begin{frame}[fragile] \frametitle{Getting started - \ixia~Lab Setup}
\begin{center}
	 \pgfuseimage{solarispcixia}
\end{center}
\begin{itemize}
	\item SunOS 5.10 running in sjc-cde-006
	\item PC running Windows in 172.27.241.81
	\item \ixia~chassis in 172.27.240.23
\end{itemize}
\end{frame}

\pgfdeclareimage[width=.9\textwidth]{picture}{images/picture}
\begin{frame} \frametitle{Whole picture}
	\begin{center}
		 \pgfuseimage{picture}
	\end{center}
\end{frame}

\begin{frame}[fragile] \frametitle{Getting started - Download}
\begin{itemize}
	\item Dowload\footnote{\href{http://www.ixiacom.com/support/downloads\_and\_updates/index.php}{Download and Updates page}} compatible versions\footnote{\href{http://www.ixiacom.com/support/product\_compatibility\_matrix/index.php}{Compatibility Matrix}}
	\begin{itemize}
		\item IxLoad 4.30 EA SP1 Build Number: 4.30.119.78
		\item IxOS 5.50 EA SP3 (Early Adopter) Build Number: 5.50.500.27
	\end{itemize}
	\item Make sure you have installed in chassis and in Windows PC compatible versions
	\begin{itemize}
		\item If you have multiple \ixos~and/or \ixload~versions, force the system to use the one you want (with Ixia Application Selector)
	\end{itemize}
	\item If you don't have an \ixia~account you can request for one from their web site.
\end{itemize}
\end{frame}

\begin{frame}[fragile] \frametitle{Getting started - Install}
I will suppose that you already have \ats~installed under {\scriptsize\ttfamily \$ATS\_USER\_PATH}
\begin{itemize}
	\item Install first \ixos~and then \ixload~on Windows PC and \ixia~Chassis if so needed
	\item For Solaris 10 machine
	\begin{itemize}
		\item Install \ixos~under {\scriptsize\ttfamily \$IXIA\_ATS\_FOLDER}\footnote{This variable must be created by you, see next slide for further understanding}
		\item Install \ixload~under {\scriptsize\ttfamily \$IXIA\_HOME}
	\end{itemize}
\end{itemize}
\end{frame}

\begin{frame}[fragile] \frametitle{Getting started - Install - Solaris}
\begin{block}{Your .bashrc file should look like this}
\begin{lstlisting}[language=Terminal]
IXIA_ATS_FOLDER="/auto/stg-devtest/ucosta/"

IXIA_HOME="${IXIA_ATS_FOLDER}/ixia"
IXIA_VERSION="5.50.500.27"
IXIA_RESULTS_DIR="${HOME}/results_ixia"
IXIA_LOGS_DIR="${HOME}/logs_ixia"
IXIA_TCL_DIR="${IXIA_HOME}/lib"
TCLLIBPATH="${IXIA_TCL_DIR}"

ATS_USER_PATH="${IXIA_ATS_FOLDER}/ats"
AUTOTEST="$ATS_USER_PATH"
ATS_EASY="$ATS_USER_PATH"

MANPATH="${MANTPATH}:${IXIA_HOME}/man:/usr/local/man:/usr/man:/usr/share/man:/usr/autotool/devel/man:"
PATH="${PATH}:${ATS_USER_PATH}/bin:$IXIA_HOME/bin:${ATS_USER_PATH}/man:"
export ATS_USER_PATH AUTOTEST ATS_EASY PATH MANPATH LD_LIBRARY_PATH IXIA_HOME IXIA_VERSION IXIA_RESULTS_DIR IXIA_LOGS_DIR IXIA_TCL_DIR TCLLIBPATH IXIA_ATS_FOLDER
\end{lstlisting}
\end{block}
\end{frame}

\begin{frame}[fragile] \frametitle{Getting started - Install - Solaris - part 2}
After change the .bashrc file\footnote{If you use csh as your shell, translate the previous code} don't forget to type:
\begin{lstlisting}[language=Terminal]
[ucosta@sjc-cde-006:/]-$ source $HOME/.bashrc
\end{lstlisting}
\begin{block}{If the installation of \ixload~in Solaris $10$ fails}
You can activate the debug flag and then try to understand whats wrong ({\scriptsize\ttfamily log.txt} file):
\begin{lstlisting}[language=Terminal]
[ucosta@sjc-cde-006:ixia]-$ export LAX_DEBUG=true
[ucosta@sjc-cde-006:ixia]-$ ./IxLoadTclAPI4.30.119.78 2> log.txt
\end{lstlisting}
\end{block}
\end{frame}

\begin{frame}[fragile] \frametitle{Getting started - Install - Solaris - part 3}
\begin{block}{If the installation of \ixload~in Solaris $10$ fails and you run out of patience}
You can copy my \ixload~directory into your \ixia~folder.
\begin{lstlisting}[language=Terminal]
[ucosta@sjc-cde-006:ixia]-$ cp -r /auto/stg-devtest/ucosta/ixia/IxLoadTclAPI4.30.119.78-EB $IXIA_HOME
\end{lstlisting}
\end{block}
I also have a zip file that contains this folder, if you want copy that and unzip it in your folder
\begin{block}{}
\begin{lstlisting}[language=Terminal]
[ucosta@sjc-cde-006:ixia]-$ cp /auto/stg-devtest/ucosta/ixia/IxLoad4.30EASP1.zip $IXIA_HOME
[ucosta@sjc-cde-006:ixia]-$ cd $IXIA_HOME
[ucosta@sjc-cde-006:ixia]-$ unzip IxLoad4.30EASP1.zip
\end{lstlisting}
\end{block}
\end{frame}

\begin{frame}[fragile] \frametitle{Getting started - Checking installation}
If you follow the steps you should be able to see that \ixos~and \ixload~are properly installed
\begin{block}{}
\begin{lstlisting}[language=Terminal]
[ucosta@sjc-cde-006:ixia]-$ expect
expect1.1> package require IxLoad
Tcl Client is running Ixia Software version: 5.50.500.27
4.30.119.78
expect1.2>
\end{lstlisting}
\end{block}
\end{frame}

\begin{frame}[fragile] \frametitle{\autoeasy}
\begin{block}{\autoeasy~files}
\begin{description}
	\item[CONFIG file] contains how to access our routers, passwords, etc
	\item[Job file] contains the scripts and parameters that need to be submitted for execution
	\item[Script] is where the recipe is (in this case \ats+\ixload~tests)
\end{description}
\end{block}
\end{frame}

%Task execution in single process (-on_proc)
%This execution mode is similar to the synchronous mode with the exception that all tasks execute within the same UNIX process. A unique 
%string (process name) is associated with the process that was created to execute the task remains so that future programs can be run on 
%the same process and have access to the same Tcl data structures.
%ats_run -on_proc procname -tid taskid1 task1 parameters1
\begin{frame}[fragile] \frametitle{\ats~files - .job and CONFIG files}
\begin{block}{CONFIG file}
\begin{lstlisting}[language=Terminal]
#activate ATS debug
set LOG_LEVEL { 
    aereport debug
}
set REPORTS ucosta@cisco.com
set TESTBEDS {ucosta_router_tb}
set ROUTERS(ucosta_router_tb) {ucosta_router}

global _device
set _device(ucosta_router) "telnet 172.19.218.32 2013"
TacacsPw {}
EnablePw {}

\end{lstlisting}
\end{block}
\begin{block}{run.job file}
\begin{lstlisting}[language=Terminal]
ats_run -on_proc abc123 test.ixload test.ixload 1 ixia DEBUG  172.27.240.23 "1,1,9  1,1,10" 100 full 172.27.241.81
\end{lstlisting}
\end{block}
\end{frame}

\begin{frame}[fragile] \frametitle{\ats~files - Makefile(optional)}
Makefile to automate the process of run the test scripts
\begin{description}
	\item[make run\_log] works only in Solaris machine
	\item[make watch] its good to watch the output that is being generated
	\item[make run] run the test scripts and send the output to sdtout
	\item[make clean] keep our dir clean
\end{description}
\begin{block}{makefile}
\begin{lstlisting}[language=Terminal]
run_log:run.job CONFIG
    autoeasy -D run.job -cf CONFIG > log
watch:log
    watch -n 1 'cat log | tail -n 30'
run:run.job CONFIG
    autoeasy -D run.job -cf CONFIG
clean:
    rm -f *~ *.*~ *.log *.report *.rerun log
\end{lstlisting}
\end{block}
\end{frame}

\begin{frame}[fragile] \frametitle{\ixload+\ats~script}
For now we will use the GSBU Dev Test team framework\footnote{Can be found in regression/tests/functionality/gsg/ after you checkout the most recent version of regression tests}\\
We will use as example a generation of \http~traffic.
\begin{block}{Structure of the script}
\begin{lstlisting}[language=TclUlisses]
<imports>
<parse args>
test_config { ... }
test_analyze { ... }
test_unconfig { ... }
\end{lstlisting}
\end{block}
\end{frame}

\begin{frame}[fragile] \frametitle{\ixload+\ats~script - test\_config}
\begin{block}{Structure of the script}
\begin{lstlisting}[language=TclUlisses]
test_config {
	tg-ixiaLoad_connect $PCServerIP $tgArgs
	tg-ixiaLoad_client_net -port $tgPort2 -firstIp $IxLoadClientIP  -firstMac 00:C6:12:02:01:00 -ipCount 1  -networkMask $netmask -gateway 172.31.254.254 #configure client network
	tg-ixiaLoad_server_net -port $tgPort1 -firstIp $IxLoadServerIP  -firstMac 00:B6:12:02:01:00 -ipCount 1  -networkMask $netmask -gateway 172.16.254.254 #configure server network
	tg-ixiaLoad_client_http_traffic -maxSessions 1 -pageList $pageList -httpVersion 1.1 #configure client
	tg-ixiaLoad_server_http_traffic -httpPort 80 #configure server
}
\end{lstlisting}
\end{block}
\end{frame}

\begin{frame}[fragile] \frametitle{\ixload+\ats~script - test\_config - part 2}
\begin{block}{Structure of the script - cont.}
\begin{lstlisting}[language=TclUlisses]
test_config {
	tg-ixiaLoad_client_traf_net_map -objectiveType concurrentConnections -objectiveValue 2000 -iterations 1 -rampDownTime 10  -sustainTime 20 #configure client traffic
	tg-ixiaLoad_server_traf_net_map #configure server traffic
	tg-ixiaLoad_create_test #create test
}
\end{lstlisting}
\end{block}
\end{frame}

\begin{frame}[fragile] \frametitle{\ixload+\ats~script - test\_analyze}
\begin{block}{Structure of the script - cont.}
\begin{lstlisting}[language=TclUlisses]
test_analyze {
	set ixLoadStats tg-ixiaLoad_run_test_with_stats #run HTTP test and get IxLoad stats
}
\end{lstlisting}
\end{block}
The result is given in the form of HashTable, we can access it by:
\begin{lstlisting}[language=TclUlisses]
set clientBytesReceived [keylget ixLoadStats client,BytesReceived]
set clientBytesReceived [double $clientBytesReceived]
set serverBytesSent     [keylget ixLoadStats server,BytesSent]
set serverBytesSent     [double $serverBytesSent]
echo "--- clientPacketsReceived = $clientPacketsReceived"
echo "--- serverPacketsSent     = $serverPacketsSent"
\end{lstlisting}
\end{frame}

\begin{frame} \frametitle{Demo}
	\begin{center}\huge{Demo}\end{center}
\end{frame}

\begin{frame} \frametitle{Questions}
	\begin{center}\huge{?}\end{center}
\end{frame}

\end{document}
