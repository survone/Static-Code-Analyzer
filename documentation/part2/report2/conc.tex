\chapter{Conclusão e Trabalho Futuro}\label{chap con} 
Ao longo de toda primeira fase modelamos os vários aspectos do nosso sistema. Nesta segunda fase partimos para a implementação do
sistema e focamos também a nossa atenção na exploração de um frontend para a linguagem C.\\
\\
Toda a modelação do sistema realizada, foi importante, devido à visão mais alargada que nos deu do problema e da sua resolução.
A implementação correu de forma tranquila, onde apenas pequenos ajustes se realizaram, em relação ao pensado na fase anterior.
O trabalho relativo à importação de enunciados e tentativas em formato XML, que tinha ficado incompleto na fase anterior, foi agora 
completado.\\
O sistema está no fim desta fase apto a receber as soluções dos utilizadores, tratá-las e apresentar resultados.\\
Foi também dado inicio ao desenvolvimento de uma interface em linha de comandos escrita em Perl, que alarga a acessibilidade à
aplicação, deixando do acesso à mesma estar dependente de um browser.\\
A exploração do frontend \textit{Language.C} também já deu os primeiros passos, sendo o objectivo final da sua utilização a geração
de métricas sobre o código fonte submetido.\\
\\
Cremos que no fim desta fase, o desenvolvimento da aplicação web está próximo do fim, restando ainda corrigir eventuais bugs, adicionar
algumas funcionalidades e tornar a interface mais apelativa e intuitiva. Algumas destas funcionalidades como a apresentação
de estatísticas relativas ao número de ficheiros por linguagem/linhas de código  e detecção de clones, estão já numa fase avançada de
desenvolvimento, faltando a sua integração com o sistema.\\
O trabalho à volta da interface em linha de comandos e do frontend \textit{Language.C} irá continuar, com vista a que os objectivos finais
sejam cumpridos.
