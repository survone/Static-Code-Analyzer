\chapter{Front-end}
\minitoc
Como já tinhamos explicado na primeira milestone, decidimos que o nosso sistema vai tentar suportar ao máximo avaliação de métricas sobre código C.
Seria muito interessante suportar outras, mas acreditamos nesta altura que se o nosso sistema for extendido para suportar a avaliação de outras linguagens que não o C
então deveriamos recorrer a ferramentas externas que fizessem algum trabalho por nós.\\
Relativamente ao Front-end que utilizamos, ele está feito em Haskell e foi um GSoc (Google Summer of Code) feito em 2008, chama-se Language.C\footnote{Mais informação em: \url{http://trac.sivity.net/language\_c}}.
Este pacote de software apresenta um completo e bem testado parser e pretty printer para a definição da linguagem \textrm{C99} e ainda muitas das \textrm{GNU extensions}.\\

A nossa ideia é pegar em toda a investigação e trabalho dedicado à análise e descoberta de métricas, que estão descritas no Capítulo \ref{chap:metricas}, e implementa-las
utilizando este Front-end.\\

Inicialmente decidimos partir para a exploração da linguagem (dos tipos de dados) que estavam definidos neste parser. Rápidamente encontramos a AST da linguagem \textrm{C99}
e assim descobrimos que a linguagem C não é assim tão grande como estariamos à espera, como podemos ver no Apêndice \ref{chap:ast}.

\section{Estudo do Front-End}
Todos os tipos deste parser estão munidos de um \textrm{NodeInfo}, que nada mais é do que a informação relativa ao ficheiro, número de linha e coluna onde apareceu
esta derivação.\\
Um ficheiro em linguagem \textrm{C99} é representado como uma lista de declarações externas que pode ser uma declaração ou uma definição de função como podemos ver
na secção \ref{chap:extdefin}.

\begin{haskell}
data CTranslUnit = CTranslUnit [CExtDecl] NodeInfo
data CExtDecl = CDeclExt CDecl
              | CFDefExt CFunDef
              | CAsmExt CStrLit
\end{haskell}

Depois podemos ver que uma declaração externa pode ser então uma declaração como podemos ver na secção \ref{chap:decl}, vejamos o que é isto de declaração no \textrm{Haskell}:
\begin{haskell}
data CDecl = CDecl [CDeclSpec] [(Maybe CDeclr, Maybe CInit, Maybe CExpr)] NodeInfo
\end{haskell}
Este tipo de declarações é bastante abrangente e inclui declarações de estruturas de dados, declaração de parâmetros e tipos de dados.\\

Tal como  a definição é uma lista de declarações C e qualificadores:
\begin{haskell}
data CDeclSpec = CStorageSpec CStorageSpec  -- ^ storage-class specifier or typedef
               | CTypeSpec    CTypeSpec     -- ^ type name
               | CTypeQual    CTypeQual     -- ^ type qualifier

data CStorageSpec = CAuto     NodeInfo     -- ^ auto
                  | CRegister NodeInfo     -- ^ register
                  | CStatic   NodeInfo     -- ^ static
                  | CExtern   NodeInfo     -- ^ extern
                  | CTypedef  NodeInfo     -- ^ typedef
                  | CThread   NodeInfo     -- ^ GNUC thread local storage

data CTypeSpec = CVoidType    NodeInfo
               | CCharType    NodeInfo
               | CShortType   NodeInfo
               | CIntType     NodeInfo
               | CLongType    NodeInfo
               | CFloatType   NodeInfo
               | CDoubleType  NodeInfo
               | CSignedType  NodeInfo
               | CUnsigType   NodeInfo
               | CBoolType    NodeInfo
               | CComplexType NodeInfo
               | CSUType      CStructUnion NodeInfo  -- ^ Struct or Union specifier
               | CEnumType    CEnum        NodeInfo  -- ^ Enumeration specifier
               | CTypeDef     Ident        NodeInfo  -- ^ Typedef name
               | CTypeOfExpr  CExpr        NodeInfo  -- ^ @typeof(expr)@
               | CTypeOfType  CDecl        NodeInfo  -- ^ @typeof(type)@
\end{haskell}

\textrm{CTypeQual} diz respeito a qualificadores de tipos, por exemplo:
\begin{itemize}
\item const
\item volatile
\item restrict
\item inline
\item \_\_attribute\_\_
\end{itemize}

\begin{haskell}
data CTypeQual = CConstQual NodeInfo
               | CVolatQual NodeInfo
               | CRestrQual NodeInfo
               | CInlineQual NodeInfo
               | CAttrQual  CAttr
\end{haskell}

E ainda uma lista de triplos de \textrm{Maybe}s (Maybe CDeclr, Maybe CInit, Maybe CExpr).
Esta lista como tem 3 Maybes pode apresentar várias formas, estas formas definem o tipo de declaração que temos:

\begin{description}
\item[Toplevel declarations] Uma declaração na forma (Just declr, init?, Nothing)
\item[Structure declarations] Declarações na forma (Just declr, Nothing, size?)
\item[Parameter declarations] Declarações na forma (Just declr, Nothing, Nothing) ou (Nothing, Nothing, Just size)
\item[Toplevel declarations] Declarações na forma (Just declr, Nothing, Nothing)
\end{description}

Tempo agora para mostrar o que são estas definições \textrm{CDeclr}, \textrm{CInit} e \textrm{CExpr}.

\begin{haskell}
data CDeclr = CDeclr (Maybe Ident) [CDerivedDeclr] (Maybe CStrLit) [CAttr] NodeInfo
\end{haskell}
\textrm{CDeclr} corresponde a declarações de funções ou tipos em C. Por exemplo:

\begin{code_files}
int x;
CDeclr "x" []
\end{code_files}

\begin{code_files}
const int * const * restrict x;
CDeclr "x" [CPtrDeclr [restrict], CPtrDeclr [const]]
\end{code_files}

\begin{code_files}
int* const f();
CDeclr "f" [CFunDeclr [],CPtrDeclr [const]]
\end{code_files}

Ou seja, primeiro temos um possivel identificador, seguido de uma lista de 
declaraçõe derivadas seguido de um string literal.

Vejamos agora a definição de \textrm{CDerivedDeclr}:

\begin{haskell}
data CDerivedDeclr = CPtrDeclr [CTypeQual] NodeInfo
                   -- ^ Pointer declarator @CPtrDeclr tyquals declr@
                   | CArrDeclr [CTypeQual] (CArrSize) NodeInfo
                   -- ^ Array declarator @CArrDeclr declr tyquals size-expr?@
                   | CFunDeclr (Either [Ident] ([CDecl],Bool)) [CAttr] NodeInfo
                   -- ^ Function declarator @CFunDeclr declr (old-style-params | new-style-params) c-attrs@
\end{haskell}

Isto basicamente corresponde a uma atribuição de um pointer (CPtrDeclr attrs),
uma atribuição do resultado da computação de uma função (CFunDeclr attrs) e
atribuição de um array de elementos (CArrayDeclr attrs).

\textrm{CAttr} é simplesmente a utilização da anotação \_\_attribute\_\_

Voltando ao triplo de Maybes, temos ainda como segundo elemento \textrm{CInit} que correpondem a inicializações em C.

\begin{haskell}
data CInit = CInitExpr CExpr
             NodeInfo            -- ^ assignment expression
           | CInitList CInitList
             NodeInfo            -- ^ initialization list (see 'CInitList')
\end{haskell}
Estes inicializadoes podem ser uma de duas coisas, ou um assignment a uma expressao ou então
uma inicialização de uma lista, rodeada de parêntices de bloco.\\
Usemos então o caso das inicializações para introduzir as \textrm{CExpr}.

\begin{haskell}
data CExpr = CComma       [CExpr]       -- comma expression list, n >= 2
             NodeInfo
           | CAssign      CAssignOp     -- assignment operator
             CExpr         -- l-value
             CExpr         -- r-value
             NodeInfo
           | CCond        CExpr         -- conditional
             NodeInfo
           | CBinary      CBinaryOp     -- binary operator
             CExpr         -- lhs
             CExpr         -- rhs
             NodeInfo
           | CCast        CDecl         -- type name
             CExpr
             NodeInfo
           | CUnary       CUnaryOp      -- unary operator
             CExpr
             NodeInfo
           | CSizeofExpr  CExpr
             NodeInfo
           | CSizeofType  CDecl         -- type name
             NodeInfo
           | CAlignofExpr CExpr
             NodeInfo
           | CAlignofType CDecl         -- type name
             NodeInfo
           | CComplexReal CExpr         -- real part of complex number
             NodeInfo
           | CComplexImag CExpr         -- imaginary part of complex number
             NodeInfo
           | CIndex       CExpr         -- array
             CExpr         -- index
             NodeInfo
           | CCall        CExpr         -- function
             [CExpr]       -- arguments
             NodeInfo
           | CMember      CExpr         -- structure
             Ident         -- member name
             Bool          -- deref structure? (True for `->')
             NodeInfo
           | CVar         Ident         -- identifier (incl. enumeration const)
             NodeInfo
           | CConst       CConst        -- ^ integer, character, floating point and string constants
           | CCompoundLit CDecl
             CInitList     -- type name & initialiser list
             NodeInfo      -- ^ C99 compound literal
           | CStatExpr    CStat NodeInfo  -- ^ GNU C compound statement as expr
           | CLabAddrExpr Ident NodeInfo  -- ^ GNU C address of label
           | CBuiltinExpr CBuiltin        -- ^ builtin expressions, see 'CBuiltin'
\end{haskell}
Esta definição é tão completa que abrabge também algumas extenções feitas à linguagem C pelo, conhecidas como
GCC Extensions, como por exemplo as: alignof, \_\_real, \_\_imag, (\{ stmt-expr \}).

Terminada a explicação extensa de \textrm{CDecl}, continuamos a explicação de \textrm{CExtDecl} e passamos à definição de
\textrm{CFunDef}, ou seja uma definição de função como podemos ver em \textrm{function-definition} na secção \ref{chap:extdefin}.

\begin{haskell}
data CFunDef = CFunDef [CDeclSpec]      -- type specifier and qualifier
               CDeclr           -- declarator
               [CDecl]          -- optional declaration list
               CStat            -- compound statement
               NodeInfo
\end{haskell}

Esta definição usa uma lista de especificadores de tipo e de quantificação, uma declaração,
uma lista adicional de declarações e um C Statement. Como já explicamos todos, passamos para a explicação
do \textrm{CStat}.\\

Esta definição é capaz de ser a mais familiar e fácil de perceber na medida em que é aqui que estão
definidos os statements C que caracterizam esta linguagem:

\begin{haskell}
data CStat = CLabel  Ident CStat [CAttr] NodeInfo  -- ^ An (attributed) label followed by a statement
           | CCase CExpr CStat NodeInfo            -- ^ A statement of the form @case expr : stmt@
           | CCases CExpr CExpr CStat NodeInfo     -- ^ A case range of the form @case lower ... upper : stmt@
           | CDefault CStat NodeInfo               -- ^ The default case @default : stmt@
           | CExpr (Maybe CExpr) NodeInfo
             -- ^ A simple statement, that is in C: evaluating an expression with side-effects
             --   and discarding the result.
           | CCompound [Ident] [CBlockItem] NodeInfo    -- ^ compound statement @CCompound localLabels blockItems at@
           | CIf CExpr CStat (Maybe CStat) NodeInfo     -- ^ conditional statement @CIf ifExpr thenStmt maybeElseStmt at@
           | CSwitch CExpr CStat NodeInfo
             -- ^ switch statement @CSwitch selectorExpr switchStmt@, where @switchStmt@ usually includes
             -- /case/, /break/ and /default/ statements
           | CWhile CExpr CStat Bool NodeInfo      -- ^ while or do-while statement @CWhile guard stmt isDoWhile at@
           | CFor (Either (Maybe CExpr) CDecl)
             (Maybe CExpr)
             (Maybe CExpr)
             CStat
             NodeInfo
             -- ^ for statement @CFor init expr-2 expr-3 stmt@, where @init@ is either a declaration or
             -- initializing expression
           | CGoto Ident NodeInfo            -- ^ goto statement @CGoto label@
           | CGotoPtr CExpr NodeInfo         -- ^ computed goto @CGotoPtr labelExpr@
           | CCont NodeInfo                  -- ^ continue statement
           | CBreak    NodeInfo              -- ^ break statement
           | CReturn (Maybe CExpr)NodeInfo   -- ^ return statement @CReturn returnExpr@
           | CAsm CAsmStmt NodeInfo          -- ^ assembly statement
\end{haskell}

Aqui podemos então ver as Labels, Cases, statements de default dos Cases, expressões, ifs, switches,
whiles, fors, gotos, continues, breaks, retuns e instruções assembly.\\

Grande parte deste Front-End, Language.C, fica assim explicado. Achamos que explicar os detalhes
sobre o \textrm{CStat} não seria necessáario, visto já termos detalhado os mais importantes.

\section{Bug no \textrm{Language.C} em \textrm{MacOSX}}
Depois de explorar e usar, sempre com exemplos pequenos este front-end descobrimos que se os ficheiros C submetidos fizerem uso de biliotecas externas teriamos um
problema ao correr o avaliador de software em MacOSX. Este problema já é conhecido pela equipa que desenvolve o Language.C\footnote{Conforme se pode ver neste ticket: \url{http://trac.sivity.net/language_c/ticket/2}}.\\
Embora o nosso sistema final corra numa máquina com Linux, o desenvolvimento desta aplicação de avaliação será feito em MacOSX e assim isto confere um problema nesta fase.
Mesmo assim também foi interessante descorbrir este bug porque pretendemos que a nossa aplicação corra no máximo de plataformas possíveis. Imagine-se o caso em que
queremos rápidamente instalar numa máquina o nosso sistema para demonstração ou montar um concurso rápido, sem que seja necessário um servidor.\\

Este bug prende-se com o facto do Language.C na sua versão mais recente, 3.2.1, não suportar a notação de \_\_BLOCKS\_\_ que as librarias do MacOSX têem.\\
Imagine-se o seguinte código Haskell a usar Language.C:

\begin{haskell}
module Main where

import Language.C
import Language.C.System.GCC
import Language.C.Data.Ident
import System.Environment

process :: String -> IO ()
process file = do
	stream <- parseCFile (newGCC "gcc") Nothing [] file
	case stream of
		( Left error  ) -> print error
		( Right cprog ) -> print "OK"

main :: IO ()
main = do
	files <- getArgs
	mapM_ process files
\end{haskell}

O comportamento deste programa seria receber $n$ ficheiros e a cada um deles passar ao preprocessador do GCC, seguidamente é construída a árvore de parsing que fica
guardada em $stream$ e posteriormente verificamos se tivemos algum erro, se sim reportamos o erro para o $stdout$ se não então imprimimos "OK" para o $stdout$.\\

Ao alimentar este programa com o seguinte pedaço de código:

\begin{haskell}
#include <stdlib.h>

void main(int argc, char **argv) {
	malloc(10);
	return 0;
}
\end{haskell}

acontece o seguinte erro:

\begin{code_files}
[ulissesaraujocosta@maclisses:Parser]-$ ./main main.c
/usr/include/stdlib.h:272: (column 20) [ERROR]  >>> Syntax Error !
  Syntax error !
  The symbol `^' does not fit here.
\end{code_files}

Ao inspeccionar a linha 272 do stdlib.h verificamos a presença do seguinte código:

\begin{haskell}
#ifdef __BLOCKS__
int  atexit_b(void (^)(void));
void    *bsearch_b(const void *, const void *, size_t,
size_t, int (^)(const void *, const void *));
#endif /* __BLOCKS__ */
\end{haskell}

O problema reside exactamente aqui, ou seja o Language.C não consegue descobrir esta notação. Este problema parece já ter sido aberto há alguns meses e está previsto
para correcção na próxima versão 4.0 do Language.C.\\

Afim de tentarmos mesmo assim conseguir usar este front-end lembramos-nos de proibir o preprocessador do GCC para definir a variável \_\_BLOCKS\_\_. Assim sendo
temos que incluír no nosso código a injecção de uma linha de código que fará com que todos os ficheiro C processados pelo nosso programa apresentem a desactivação desta variável:

\begin{haskell}
#undef __BLOCKS__
#include <stdlib.h>

void main(int argc) {
	malloc(10);
	return 0;
}
\end{haskell}

