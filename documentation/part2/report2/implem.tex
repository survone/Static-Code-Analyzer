\chapter{Implementação} \label{chap implem}

Neste capítulo vamos expôr alguns pormenores relacionados com a implementação do problema proposto.

\section{Criação de grupos, docentes e concorrenctes}\label{sec gdc}
O sistema permite que qualquer utilizador não registado se registe como grupo, e associe a si, um ou mais concorrentes. Este login
é utilizado por todo o grupo, para participar nos mais variados concursos.\\
As contas de docente só podem ser criados pelo administrador. Para simplificar o trabalho do administrador, o docente pode criar 
uma conta de grupo, à qual mais tarde será concedida privilégios de docente.



\section{Linguagens de programação}\label{sec lps}
O nosso sistema é multilingue, ou seja, é possível submeter código fonte em várias linguagens de programação diferentes, desde que 
a linguagem tenha sido correctamente configurada por um docente.
Cada linguagem é caracterizada por uma séria de campos, os quais serão explicados de seguida:
\begin{itemize}
\item string de compilação: string que será executada quando se pretender compilar determinado código fonte. Esta string tem a
particularidade de no lugar em que é suposto conter o nome do ficheiro a compilar, contém \textit{\#\{file\}}.\\
Desta forma a string de compilação torna-se genérica, e idependente do nome do ficheiro a compilar.
exemplo: gcc -O2 -Wall \#\{file\}

\item string simples de execução: string utilizada para executar quando o código fonte foi compilado pela string de compilação.\\
exemplo 1:\\ 
- string de compilação: gcc -O2 -Wall \#\{file\}\\
- string de execução respectiva: ./a.out\\
exemplo 2:\\
- string de compilação: gcc -O2 -Wall \#\{file\} -o exec\\
- string de execução respectiva: ./exec\\

\item string complexa de execução: a necessidade de uma segunda string de execução surgiu quando tentamos preparar o sistema para receber makefiles (inicialmente apenas para C). Nestes casos, o nome do executável gerado pela compilação não é conhecido à partida.
Desta forma é necessário analisar o makefile, e só depois executar, tendo em conta a informação que retiramos do makefile.\\
Assim, e para a linguagem C, a string complexa de execução seria:\\
- ./\#\{file\}\\
em que \#\{file\} representa o nome do executável.

\end{itemize}


\section{Compilação}\label{sec comp}

Estando as linguagens de programação correctamente configuradas, a compilação torna-se bastante simples. Quando uma tentativa
é submetida no sistema, começamos por verificar se foi submetida apenas um ficheiro de código, ou um ficheiro comprimido.\\
Caso seja apenas um ficheiro, o sistema tenta compilar o código submetido, com a string de compilação da linguagem de programação em causa.\\
No caso de se tratar de um ficheiro comprimido, depois de o descomprimir, o sistema verifica se existe um makefile entre os ficheiros extraídos. Caso se verifique, é corrido o comando \textit{make}, e tenta retirar o nome do executável gerado, de mode a poder
ser usado na execução.

\section{Execução}\label{sec exec}
No fim da compilação, o sistema vai executar o programa uma vez para cada input. O processo de execução no caso de a compilação ter
sido feita à custa do makefile, é feita usando a string complexa de execução (sendo o nome do executável aquele que foi retirado do
makefile) . Se tal não tiver acontecido, é usada a string simples.\\
A execução pode ser abortada se ultrapassar o tempo máximo de execução, que é definido aquando da criação do enunciado em
questão.

\section{Guardar resultados}\label{sec res}
Para cada input do enunciado em questão, o programa é executado uma vez. O seu output é comparado com o output esperado e 
é guardada uma entrada na base de dados com a percentagem de testes nos quais o programa teve sucesso.\\
No caso de o código não compilar, ou da execução do programa demorar mais tempo do que o máximo previsto pelo docente quando 
criou o enunciado, estas informações são também guardadas na base de dados.\\
Além de se guardarem todas as tentativas, a melhor é também guardada numa tabela à parte, para que o melhor resultado para cada
enunciado seja de fácil acesso.

