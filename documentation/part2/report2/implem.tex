\chapter{Implementação} \label{chap implem}

Neste capítulo vamos expôr alguns pormenores relacionados com a implementação do problema proposto.

\section{Linguagens de programação}\label{sec lps}
O nosso sistema é multilingue, ou seja, é possível submeter código fonte em várias linguagens de programação diferentes, desde que 
a linguagem tenha sido correctamente configurada por um docente.
Cada linguagem é caracterizada por uma séria de campos, os quais serão explicados de seguida:
\begin{itemize}
\item string de compilação : string que será executada quando se pretender compilar determinado código fonte. Esta string tem a
particularidade de no lugar em que é suposto conter o nome do ficheiro a compilar, contém \textit{#{file}}.\\
Desta forma a string de compilação torna-se genérica, e idependente do nome do ficheiro a compilar.
exemplo: gcc -O2 -Wall #{file}

\item string simples de execução

\end{itemize}


\section{Compilação}\label{sec comp}

Para começarmos a explicar como se procede à compilação do código submetido, quando tal é necessário, temos primeiro de explicar que o
nosso sistema permite a 