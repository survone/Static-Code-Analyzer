\chapter{Instalação}\label{chap scriptinst}
\minitoc
Após um dos elementos do grupo ter sofrido um grande infortúnio com a sua máquina de trabalho, surgiu a ideia, que deveria ter sido lembrada no início do desenvolvimento deste projecto, 
de implementar uma \emph{script} de instalação (em \emph{bash}) que cuidasse de pôr em funcionamento todo o tipo de aplicações que o projecto oferece.\\

Assim, começou-se por identificar as diversas ferramentas usadas ao longo deste projecto e que serão essenciais para o seu funcionamento final. Desde da aplicação web, com \emph{Ruby} e 
\emph{Rails} (e diversas \emph{gems} de cada), ao \emph{parser}, com \emph{Haskell}, \emph{Strafunski} e \emph{Language.C}, a passar pela interface pelo terminal implementada em 
\emph{Perl} (e os seus inúmeros módulos), vasto será o leque de preocupações que se terá que ter para que a partir de uma simples \emph{script} se ponha o projecto pronto a funcionar.\\

Neste capítulo será explicada o funcionamento da \emph{script}, desde da identificação inicial da máquina em que se encontra, à instalação do mais pequeno módulo de \emph{Perl} utilizado.
Vale a pena também referir a importância desta \emph{script}, pois além de ser bastante útil para a gestão e manutenção deste projecto, e apesar de ser uma tarefa árdua (tendo em conta 
que se começou a meio do projecto), torna-se muito gratificante pelos ganhos de \emph{skill} em administração de sistemas.\\

\section{User check}
Antes de se dar início à instalação da máquina é chamada uma função \texttt{check\_user\_id} para verificar se o utilizador é administrador da sua máquina. O seguinte pedaço de código 
diz respeito a essa função.

\begin{myxml}
 function check_user_id {
        if [ ! "`whoami`" = "root" ]; then
                echo "Not running as root. Yes this is an installation file..."
                exit 1 ;
        fi
}
\end{myxml}


\section{Identificação da máquina}\label{sec idmaq}

A \emph{script} começa a funcionar pela operação mais básica possível, a identificação da máquina em que corre. Esta poderá ter estar instalada com diversos sistemas operativos, o que 
só por si, diferenciará muito o comportamento da \emph{script}. No momento em que corre o projecto, apenas foi possível ter em consideração máquinas com sistemas operativos \texttt{Linux}
 e \texttt{MacOSX}.\\

De seguida encontra-se o pedaço de código para o funcionamento em questão:\\

\begin{myxml}
function install_package {
        case `uname -s` in
                "Darwin")
                        install_macosx
                        ;;
                "Linux")
	case `uname -v` in
		*"Ubuntu"*)
		install_ubuntu
		;;
		*)
		echo "Your Linux is not supported yet. If it does have a packet manager please send an email to $\$$admin_email"
		exit 1;
		;;
		esac
		;;
                *)
		echo "Your operative system is not supported yet. Please send an email to $\$$admin_email"
		exit 1;
		;;
        esac
}
\end{myxml}

Como se pode ver, exitem duas hipóteses principais na entrada desta função. Caso o output da função \texttt{uname} (função essa que imprime dados sobre o sistema operativo
 instalado) seja igual a ``\emph{Darwin}'', dá-se início à instalação em \texttt{MacOSX} pela invocação da função \texttt{install\_macosx} (linha 4). Caso o output seja igual a 
``\texttt{Linux}'' procura-se então mais informações sobre o sistema operativo instalado, com a \emph{flag} \texttt{-v}. Das distribuições que o \texttt{Linux} oferece, apenas está 
suportada a distribuição \texttt{Ubuntu}.\\

Dependendo agora do sistema conhecido, passa-se à invocação de uma das duas seguintes funções:\\

\begin{myxml}
function install_macosx {
        echo "Working on a MacOSX machine" | $\$$andlogfile
        build_macosx
        $\$$portins gd2
        install_perl_mac
        install_perl_modules
}

function install_ubuntu {
        echo "Working on a Ubuntu machine" | $\$$andlogfile
        build_ubuntu
        $\$$aptiins libgd-dev
        install_perl_ub
        install_perl_modules
}

\end{myxml}

\section{Instalação MacOSX}
Seguindo o raciocínio da secção anterior, a instalação em \texttt{MacOSX} parte da invocação da função \texttt{install\_macosx} que contém um conjunto de operações, geralmente dedicadas 
a uma parte do projecto. 

\subsection{Bibliotecas C}
O primeiro ponto de instalação refere-se à biblioteca \texttt{gd2}, necessária para o uso de um módulo \texttt{Perl} chamado \texttt{GD}. Esse biblioteca será indispensável para se 
gerar os gráficos descritos na secção~\ref{sec geraimg}.

\subsection{Módulos Perl}\label{subsec modulosperlmacosx}
Para se proceder à instalação dos módulos \texttt{Perl} usados neste projecto, é necessário definir a lista de módulos a instalar, como se pode ver na linha 3 da função seguinte.

\begin{myxml}
 function install_perl_modules {
        perl -MCPAN -e '$\$$CPAN->{prerequisites_policy}=follow'
        local packages=(Makefile::Parser Parse::Yapp  GD GD::Graph GD::Graph::bars GD::Graph::pie Path::Class
                  Moose Term::ReadLine Term::ReadLine::Gnu Digest::SHA DBIx::Class Data::Dumper)

        if_not_exist_install_perl_modules $\$${packages[@]}
}
\end{myxml}
 
A partir desse passo, é chamada a função \texttt{if\_not\_exist\_install\_perl\_modules \${packages[@]}} (linha 6) que procede à instalação dos módulos definidos.

\begin{myxml}
 function if_not_exist_install_perl_modules {
        local packages=()

        for module in $\$$@; do
                is_perl_module_installed $\$$module
                if [ $\$$? -eq 1 ]; then
                        echo "$\$$module installed";
                else
                        echo "$\$$module not installed, installing...";
                        packages[$\$$[$\$${#packages[@]}+1]]=$\$$module;
                fi
        done;

        EXEC="cpan -if $\$${packages[@]}"
        $\$$EXEC
}

function is_perl_module_installed {
        if [ -z "`perl -M$\$$1 -e 1 2>&1`" ]; then
                return 1;
        else
                return 0;
        fi
}
\end{myxml}

Para cada módulo que a função recebe, será testada para ver se se encontra já instalada, pela função \texttt{is\_perl\_module\_installed}. Caso se encontre instalada, será removida da
lista (linha 10). No final será contruído o comando de instalação e posteriormente executado (linha 15).

\subsection{Instalação de software essencial para a ParteIII}\label{subsec parteiii}

No que diz respeito a todas as aplicações e módulos essenciais para o desenvolvimento da Parte III deste relatório, sendo eles:

\begin{itemize}
 \item \texttt{GHCi} - Compilador de \emph{Haskell}, juntamente com um interpretador da mesma linguagem.
 \item \texttt{Happy} - Um gerador de \emph{parsers} escrito em \texttt{Haskell}.
 \item \texttt{Alex} - Um gerador lexical escrito em \texttt{Haskell}.
 \item \texttt{Language.C} - Uma biblioteca do \texttt{Haskell} para análise e geração de código \texttt{C}.
\end{itemize}

O seguinte pedaço de código diz respeito à sua compilação e instalação:\\

\begin{myxml}
function build_macosx {
        is_ghc_installed
        if [ $\$$? -eq 1 ]; then
                echo "GHC is installed, I will continue..."
                is_ghc_package_installed "happy"
                if [ $\$$? -eq 0 ]; then
                        echo "Happy is not installed, I will install"
                        $\$$portins hs-happy
                fi
                is_ghc_package_installed "alex"
                if [ $\$$? -eq 0 ]; then
                        echo "Alex is not installed, I will install"
                        $\$$portins hs-alex
                fi
                is_ghc_package_installed "language"
                if [ $\$$? -eq 0 ]; then
                        echo "Language.C is not installed, I will install"
                        build_language_c
                        cd Parser/language-c-0.3.2.1/
                        runhaskell Setup.hs install
                        cd -
                fi
        else
                is_ghc_package_installed "happy"
                if [ $\$$? -eq 0 ]; then
                        echo "Happy is not installed, I will install"
                        $\$$portins hs-happy
                fi
                is_ghc_package_installed "alex"
                if [ $\$$? -eq 0 ]; then
                        echo "Alex is not installed, I will install"
                        $\$$portins hs-alex
                fi
                echo "GHC is not installed, I will do it for you."
                $\$$portins ghc
                build_language_c
        fi
}
\end{myxml}

Começa-se por verificar a instalação do compilador de \texttt{Haskell}, caso existe, verifica se existe cada uma das bibliotecas e/ou aplicações. Caso não existe, procede para a sua 
instalação.

\subsection{Ruby, Rails e Gems}\label{subsec rrgems}

A presente secção encontra-se ainda inacabada...

\begin{myxml}
 function install_rvm_and_ruby {
        # Install RVM
        #bash < <(curl -s https://rvm.beginrescueend.com/install/rvm)
        curl -s https://rvm.beginrescueend.com/install/rvm | bash
        # Install some rubies
        source "$\$$HOME/.rvm/scripts/rvm"
        rvm get head
        rvm reload
        rvm install 1.8.7
        rvm --default 1.8.7
}

function install_rails {
        #sudo apt-get install rubygems
        #sudo apt-get install libxslt-dev libxml2-dev libsqlite3-dev
        gem install rails --version 3.0.3
        cd sample_app
        bundle install
        cd -
}
\end{myxml}

\textcolor{red}{A corrigir !!}

\section{Instalação Ubuntu}
Esta secção segue o modelo da anterior mas agora em relação ao sistema operativo \texttt{Ubuntu}. Percorre todas as operações que estão descritas na secção~\ref{sec idmaq}, na 
função \texttt{install\_ubuntu}.

\paragraph{De realçar} que a instalação de \texttt{Perl} e seus módulos (ver secção~\ref{subsec modulosperlmacosx}), a instalação de \emph{software} essencial para a ParteIII
(ver secção~\ref{subsec parteiii}) e a instalação de \texttt{Ruby}, \texttt{Rails} e \texttt{Gems} (ver secção~\ref{subsec rrgems}), ocorre de maneira bastante semelhante à descrita 
nas secções anteriores, mudando apenas uma ou outra função. 

\subsection{Bibliotecas C}
A instalação de bibliotecas \texttt{C} na \emph{script} são chamadas à função \texttt{install\_aptitude\_modules}. Essa função recebe uma lista como argumentos, e para cada elemento 
verifica se se encontra instalado. Caso não esteja corre o comando da linha 8 com a bibliteca em questão.\\

\begin{myxml}
function install_aptitude_modules {
        for pkg in $\$$@; do
                dpkg -s $\$$pkg
                if [ $\$$? -eq 0 ]; then
                        echo "module $\$$pkg installed"
                else
                        echo "module $\$$pkg not installed, installing..."
                        aptitude --assume-yes install $\$$pkg
                fi
        done;
}
\end{myxml}
