\newcommand{\metricsins}{\mathbin{>\mkern-7mu\circ\mkern-9mu>}}
\newcommand{\metricscat}{\mathbin{>\mkern-7mu+\mkern-11mu>}}
\chapter{Métricas Implementadas}\label{chap:metricasapi}
\minitoc

Neste capítulo iremos falar sobre as métricas que implementamos, descrevê-las
e explicar em detalhe como estas podem ser utilizadas em conjunto com a API
que desenvolvemos para no futuro se poder extrair mais informação apartir
do trabalho que já foi realizado até aqui.

\section{Métricas calculadas}
As métricas que o nosso sistema actual permite calcular estão divididas por
grupos conforme o seu nível de actuação, assim sendo temos métricas
relativas à complexidade, métricas relativas ás linhas de ficheiros,
linhas de comentários, funções e includes.\\

Mostramos agora todas as métricas que o nosso sistema consegue calcular até agora:
\begin{itemize}
\item Grafo de includes do sistema e de cada ficheiro
\item Nr linhas de comentários (que não são pedaços de código comentados)
\item Densidade de comentários
\item Index de Mccabe
\item NLOC (nr de linhas do pretty print)
\item Nr de linhas físicas
\item Clones por bloco
\item Assinaturas de funções e nomes de funções
\end{itemize}

Todas estas métricas podem ser aplicadas tanto ao projecto total, como
a um ficheiro ou até a apenas uma única função, dependendo se faz sentido
ou não calcular essa métrica para vários níveis de detalhe.\\
Por exemplo, o index de mccabe é possível e faz sentid calcular para
os três níveis: projecto, ficheiro e função.\\

Demos especial enfoque á contagem de número de linhas de comentários, sua densidade
porque achamos que estes traduzem um bom software ou um sistema útil para o
desenvolvimento por pessoas externas.\\
Por clone por bloco temos em consideração várias linhas contíguas e verificamos
se essas linhas ocorrem em mais algum ficheiro. Isto é uma forma mais precisa
para dizer se um pedaço de código foi clonado.

\section{Tipo de dados das métricas}
De seguida mostramos as declarações em Haskell dos tipos de dados
que suportam o conjunto a que chamamos de $Metrics$.

\begin{haskell}
type Metrics = M.Map MetricName MetricValue
type MetricName = (String, Maybe FileSrc,Maybe FunctionName)
data MetricValue =
    | Num Double
    | Clone (M.Map FileDst [(Ocurrency, LineSrc, LineDst)])
    | Includes ([SystemIncludes],[Includes])
    | FunSig [FunSignature]
    | Graphviz DotFile
    | GraphvizProject DotFile																				 
\end{haskell}

Escolhemos um $Map$ como tipo principal para guardar todas as nossas
métricas por este apresentar a flexibilidade e rapidez desejada num sistema
deste tipo.\\
Como foi dito na secção anterior há metricas que faz sentido aplicar ao projecto
todo, enquanto que outras apenas a ficheiros ou at~e funções, conseguimos
essa expressividade através de um triplo cujo primeiro membro
é obrigatório e os seguintes não o são. Deixamos assim ao critério de
quem implementa as métricas se pretende aplciar essa métrica apenas ao projecto ou
a mais alguma coisa em concreto.
Este tripo será a chave n nosso $Map$, sendo que o valor é o valor da métrica
em si.

\section{API de métricas}
Mostramos agora algumas das funções mais interessantes que compõe a API
para manipulação de métricas que desenvolvemos.\\

Para criar uma métrica apartir do zero, podemos usar a seguinte função:
\begin{haskell}
(>.>) :: Metrics -> (MetricName,MetricValue) -> Metrics
m >.> (mn,mv) = 
    case M.lookup mn m of
        Nothing    -> m'
        (Just mv') -> if mv' == mv then m else m'
    where m' = M.insert mn mv m
\end{haskell}
Esta função é quase sempre usada aquando da criação de métricas novas, um
exemplo do seu uso poderia ser:
$emptyMetrics~\metricsins~(("mccbaIndex",Nothing,Nothing), Num~10)$.
Estamos assim a dizer para criar uma métrica nova cujo seu nome é
$mccabeIndex$ e aplicamos esta métrica ao pacote de software todo e ainda
dizemos que o seu valor é o número $10$.\\

Uma outra função muito interessante é a função de união de métricas.
\begin{haskell}
(>+>) :: Metrics -> Metrics -> Metrics
m1 >+> m2 = M.union m1 m2
\end{haskell}

Esta função permite-nos por exemplo, pegar numa lista de métricas e fácilmente
converter tudo numa única métrica.
\begin{haskell}
concatMetrics :: [Metrics] -> Metrics
concatMetrics = foldl (>+>) emptyMetrics
\end{haskell}

Uma outra função é o tipico $foldr$ sobre o nosso conjunto de métricas.
\begin{haskell}
foldrM :: (MetricName -> MetricValue -> c -> c) -> c -> Metrics -> c
foldrM f s = M.foldrWithKey f s
\end{haskell}

Esta função ajudou-nos bastante no caso concreto de produzir um documento \LaTeX,
visto que muito fácilmente diziamos o que pretendiamso fazer a cada uma
das métricas que estavam dentro do nosso conjunto. Mostramos agora
um pequeno exemplo extraído desse mesmo código onde extraímos a informação
da métrica que cria um ficheiro graphviz apartir dos includes.
\begin{haskell}
...
foldrM step noop m
    where step k v r =  "\\begin{dot2tex}[]"
                  >> (fromString $ fromGraphvizP v)
                  >> "\\end{dot2tex}"
                // r
          fromGraphvizP (GraphvizProject l) = l
\end{haskell}

Uma outra função muito útil para nós foi a seguinte:
\begin{haskell}
getMetricsFrom :: (a -> IO Metrics) -> [a] -> IO Metrics
getMetricsFrom f l =
    forkMapM f l >>=
        return . concatMetrics . map (either (const emptyMetrics) id)
\end{haskell}
Graças a esta função conseguimos tornar o nosso calculador muito escalável.
Esta função pode ser vista como um $map$ monadico que paraleliza as várias
computações. Ou seja, pegamos numa função e numa lista e executamos esta função
em paralelo para os vários elementos da lista.\\
Conseguimos graneds ganhos de performance graças a esta função.\\

Um exemplo do seu uso, para extraír a métrica index de mccabe de uma lista
de ficheiros.
\begin{haskell}
getListOfCFiles :: FilePath -> IO [FilePath]
getTreeFromFile :: FilePath -> [FilePath] -> IO [(FilePath,CTranslUnit)]
mccabePerFun :: (FilePath,CTranslUnit) -> IO Metrics

getListOfCFiles fp >>= getTreeFromFile fp >>= getMetricsFrom mccabePerFun
\end{haskell}

\section{Exemplos de cálculo de métricas}
Agora mostramos alguns exemplos de métricas calculadas por nós.\\
Para calcular o index de mccabe, decidimos abordar a questão não pela sua definição formal, mas sim por outras
definições alternativas que encontramos. Assim, fomos por uma solução muito simples de implementar com o
$Strafunski$ que apenas conta o número de ocorências para determinados construtores de tipos.\\
Estamos a dizer para encontrar todas as ocorrêcnias de $Ifs$, $Switches$, $Whiles$, $Fors$ e ainda operadores $and$ e $or$ e adicionar o valor
$1$ ao resultado.\\

Quando usamos esta métrica para funções apenas passamos como input uma única função de cada vez e não o programa todo. Isso
é muito fácil de obter, visto que o Language.C representa um ficheiro como uma lista de funções.

\begin{haskell}
mccabeIndex :: Data a => a -> IO Int
mccabeIndex = applyTU (full_tdTU typesOfInstr)

typesOfInstr = constTU 0
  `adhocTU` loop
  `adhocTU` binaryOp
loop :: CStat -> IO Int
loop = return . loop_
  where loop_ (CIf _ _ _ _)    = 1
        loop_ (CSwitch _ _ _)  = 1
        loop_ (CWhile _ _ _ _) = 1
        loop_ (CFor _ _ _ _ _) = 1
        loop_ _                = 0
        binaryOp :: CBinaryOp -> IO Int
        binaryOp = return . binaryOp_
          where binaryOp_ CLndOp = 1
                binaryOp_ CLorOp  = 1
                binaryOp_ _      = 0
\end{haskell}

Mostramos ainda a implementação de uma métrica bastante simples, que conta o número de linhas (pretty print) que uma determinada
função tem. Esta métrica é importante para evitar contar o número de linhas directamente do ficheiro e evitamos assim
o estilo de programação de cada programador.

\begin{haskell}
ncloc :: (FilePath,CTranslUnit) -> IO Metrics
ncloc (file,tree) =
  let len = (length . filter (not . null) . lines . show . pretty) tree
  in return $ emptyMetrics
       >.> (("ncloc",Just file,Nothing),Num $ fromIntegral len)
\end{haskell}

