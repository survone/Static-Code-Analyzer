\chapter{Strafunski}
\minitoc

Strafunski é uma biblioteca de programação genérica que implementa estratégias aplicadas ao paradigma funcional \cite{Lammel:2002:DPF:570186.570187,LV03-PADL}.\\
Esta biblioteca segue a infraestrutura de \textit{Scrap your boilerplate} \cite{SYB3} do
Haskell\footnote{Todos os módulos que estão debaixo do Data.Generics e cujos tipos implementam as classes Data e Typeable} e com isto pretende-se que o programador
escreva código mais genérico que pode funcionar sobre vários tipos de dados e ao mesmo tempo código mais curto.\\
Esta abordagem requer muita investigação e nem tanto começar de imediato a programar, visto haver bastante teoria por trás destas filosofias.

\section{Explicação}
Nesta secção iremos explciar com mais algum detalhe as funções disponíveis na API do Strafunski e o que conseguimos fazer com elas.\\
O pacote Strafunski está dividido nos seguintes módulos:
\begin{code_files}
Data
    Generics
        Strafunski
            StrategyLib
                Data.Generics.Strafunski.StrategyLib.ChaseImports
                Data.Generics.Strafunski.StrategyLib.ContainerTheme
                Data.Generics.Strafunski.StrategyLib.EffectTheme
                Data.Generics.Strafunski.StrategyLib.FixpointTheme
                Data.Generics.Strafunski.StrategyLib.FlowTheme
                Data.Generics.Strafunski.StrategyLib.KeyholeTheme
                Data.Generics.Strafunski.StrategyLib.MetricsTheme
                Data.Generics.Strafunski.StrategyLib.MonadicFunctions
                Data.Generics.Strafunski.StrategyLib.NameTheme
                Data.Generics.Strafunski.StrategyLib.OverloadingTheme
                Data.Generics.Strafunski.StrategyLib.PathTheme
                Data.Generics.Strafunski.StrategyLib.RefactoringTheme
                Data.Generics.Strafunski.StrategyLib.StrategyInfix
                Data.Generics.Strafunski.StrategyLib.StrategyPrelude
                Data.Generics.Strafunski.StrategyLib.StrategyPrimitives
                Data.Generics.Strafunski.StrategyLib.TermRep
                Data.Generics.Strafunski.StrategyLib.TraversalTheme
\end{code_files}
É interessante explicar apenas alguns destes módulos, os que conteem as funções principais e nem tanto explicar em grande detalhe cada um deles.\\

\subsubsection{Tipos de dados}
Como já foi referenciado atrás para o Strafunski conseguir processar um tipo de dados criado por nós, este tem de ser instância de Data e de Typeable.
Porque por exemplo a classe Data obriga à implementação de um catamorfismo para o nosso tipo de dados, chamado de $gfoldl$ e um anamorfismo $gunfold$, entre outras
funções para travessias na nossa árvore de tipos. Como nota é importante dizer que o utilizador (a pessoa que desenhou os tipos de dados) não necessita
fazer a instânciação manual desta classe. Podemos recorrer ao confortável \textbf{deriving Data} que o Haskell suporta para qualquer declaração
de tipos\footnote{Internamente o GHC trata de chamar a biblioteca do DrIFT. Esta faz uma análise ao código e infere as instâncias para essas classes, mais informações em: \url{http://repetae.net/computer/haskell/DrIFT/}}.\\

No que diz respeito aos tipos de dados utilizados pelo Strafunski temos dois, o \textbf{TP} para estratégias de preservação dos tipos
(o tipo do output e input coincidem), e \textbf{TU} para estratégias de tipo unificados (o output é sempre de um determinado tipo independentemente do tipo de entrada),
como é explicado em \cite{Lammel:2002:DPF:570186.570187}.\\
Estes tipo aos olhos do utilizador são abstractos, porque o lote de funções que nos são dadas para os trabalhar não implica o conhecimento destes.

\subsection{Estratégias}
De seguida iremos explicar no que consistem estas estratégias e como é que elas ganham forma no Strafunski, para efeitos de documentação iremos separar a explicação
das funções pelo pacote a que pertecem, dentro do Strafunski.\\
É de notar que para cada função que existe para o tipo de dados \textbf{TU} também existe a mesma para o \textbf{TP}.

\subsubsection{Data.Generics.Strafunski.StrategyLib.StrategyPrimitives}

Para a aplicação de uma estratégia a uma instância do nosso tipo de dados $t$ iremos usar:
$$applyTP :: (Monad~m, Term~t) \Rightarrow TP~m \rightarrow t \rightarrow m~t$$
$$applyTU :: (Monad~m, Term~t) \Rightarrow TU~u~m \rightarrow t \rightarrow m~u$$

Permite adicionar estratégias:
$$adhocTP :: (Monad~m, Term~t) \Rightarrow TP~m \rightarrow (t \rightarrow m t) \rightarrow TP~m$$
$$adhocTU :: (Monad~m, Term~t) \Rightarrow TU a m \rightarrow (t \rightarrow m u) \rightarrow TU~u~m$$

Permite executar estratégias em sequência:
$$seqTP :: Monad~m \Rightarrow TP~m \rightarrow TP~m \rightarrow TP~m$$
$$seqTU :: Monad~m \Rightarrow TP~m \rightarrow TU~u~m \rightarrow TU~u~m$$

% $$passTP :: Monad~m \Rightarrow TU~u~m \rightarrow (u \rightarrow TP~m) \rightarrow TP~m$$
% $$passTU :: Monad~m \Rightarrow TU~u~m \rightarrow (u \rightarrow TU~u~m) \rightarrow TU~u~m$$

Para tentar usar estratégias diferentes:
$$choiceTP :: MonadPlus~m \Rightarrow TP~m \rightarrow TP~m \rightarrow TP~m$$
$$choiceTU :: MonadPlus~m \Rightarrow TU~u~m \rightarrow TU~u~m \rightarrow TU~u~m$$

Aplica esta estratégia a todos os subtermos imediatos, para o \textbf{TU} os resultados são reduzidos com a função do monoid $+$:
$$allTP :: Monad~m \Rightarrow TP~m \rightarrow TP~m$$
$$allTU :: (Monad~m, Monoid~u) \Rightarrow TU~u~m \rightarrow TU~u~m$$

\subsubsection{Data.Generics.Strafunski.StrategyLib.TraversalTheme}

Esta função permite aplicar a estratégia ao primeiro filho que aparecer da esquerda para a direita:
$$once\_tdTP,once\_buTP :: MonadPlus~m \Rightarrow TP~m \rightarrow TP~m$$
$$once\_tdTU, once\_buTU :: MonadPlus~m \Rightarrow TU~u~m \rightarrow TU~u~m$$

\subsubsection{Data.Generics.Strafunski.StrategyLib.StrategyPrelude}

Função identidade:
$$idTP :: Monad~m \Rightarrow TP~m$$

Função constante:
$$constTU :: Monad~m \Rightarrow u \rightarrow TU~u~m$$

A seguinte função é uma estratégia em si que falha sempre:
$$failTP :: MonadPlus~m \Rightarrow TP~m$$
$$failTU :: MonadPlus~m \Rightarrow TU~u~m$$

\section{Exemplos}
Agora iremos mostrar alguns exemplos de como usar a biblioteca Strafunski para conseguir processar a nossa árvore de parsing.\\

\subsection{McCabe Index$--$}
O índice de McCabe consiste, como já referimos anteriormente, num grafo orientado onde os nodos são as computações. Para se chegar a este número
temos de aplicar a formúla já descrita. Uma forma incipiente para calcular o índice de McCabe pdoeria ser contar o número de ifs, switches, fors e mais algumas
expressões de controlo. Assim conseguimos fazer isto com o Strafunski da seguinte forma:
\begin{haskell}
testMcCabe :: IO Int 
testMcCabe =  parr >>= mcCabeIndex . fromRight

instance Num a => Monoid a where
    mappend = (+) 
    mempty = 0 
        
mcCabeIndex :: Data a => a -> IO Int 
mcCabeIndex =  applyTU (full_tdTU isConditional)
        
isConditional = constTU 0 `adhocTU` (return . test)

test (CIf _ _ _ _)    = 1
test (CSwitch _ _ _)  = 1
test (CWhile _ _ _ _) = 1
test (CFor _ _ _ _ _) = 1
test _                = 0
\end{haskell}
Como já explicamos anteriormente o tipo \textbf{TU} quando usado com determinados operadores exige que tenhamos uma instância de \textbf{Monoid} para o tipo de retorno.
Assim fomos obrigados a definir a instância de \textbf{Monoid} para os tipos Numerários do Haskell como a função soma. Temos consciência de que isto por si só
é uma grande restrição e também temos consciência que existem alternativas para resolver este problema, como é o caso das classes \textbf{Sum} e \textbf{Mul}.
No entanto não conseguimos pôr a funcionar a tempo de entrega deste relatório essas instâncias com o tipo \textbf{TU} do Strafunski.

\subsection{Extracção das assinaturas das funções}
Um outro exemplo muito interessante e bastante útil é a extracção do cabeçalho das funções em C que constam no nosso código. Isto é bastante útil para se ter uma noção
do tamanho da implementação e da sua estruturação.

\begin{haskell}
getFunctionsSign :: IO [CExtDecl]
getFunctionsSign = parr >>= return . getFunSign . fromRight

getFunSign = applyTU (once_tdTU names1)
names1 = failTU `adhocTU` fromFunctionToSign

fromFunctionToSign (CFDefExt (CFunDef lCDeclSpec cDeclr _ _ nInfo ))
    = [CDeclExt (CDecl lCDeclSpec [(Just $ cDeclr,Nothing,Nothing)] internalNode)]
fromFunctionToSign _ = []
\end{haskell}

Como podemos ver na função $fromFunctionToSign$ apenas estamos a converter de um construtor do tipo \textbf{CExtDecl}, que é o \textbf{CFDefExt} para o mesmo tipo,
mas com o contrutor \textbf{CDeclExt}, ou seja estamos a passar o nodo da árvore de uma definição para uma declaração.\\

Para o caso do seguinte código:
\begin{code_files}
void main(int argc, char **argv) {    
    if(1) {}                                      
    switch (1) {                  
        case 1: {    
            lala(1,1);    
            break;                                 
        }    
        case 1: {                      
            break;    
        }                                                                         
    }                    
    return 0;    
}                                                                                         

void lala(int a, int b) {    
    if(a) {}    
}  
\end{code_files}
iriamos ter o seguinte output:
\begin{code_files}
*Main> getFunctionsSign >>= putStrLn . unlines . map (show . pretty)
void main(int argc, char * * argv);
void lala(int a, int b);
\end{code_files}
